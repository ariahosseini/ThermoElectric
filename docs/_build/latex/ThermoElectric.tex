%% Generated by Sphinx.
\def\sphinxdocclass{report}
\documentclass[letterpaper,10pt,english]{sphinxmanual}
\ifdefined\pdfpxdimen
   \let\sphinxpxdimen\pdfpxdimen\else\newdimen\sphinxpxdimen
\fi \sphinxpxdimen=.75bp\relax
\ifdefined\pdfimageresolution
    \pdfimageresolution= \numexpr \dimexpr1in\relax/\sphinxpxdimen\relax
\fi
%% let collapsible pdf bookmarks panel have high depth per default
\PassOptionsToPackage{bookmarksdepth=5}{hyperref}

\PassOptionsToPackage{warn}{textcomp}
\usepackage[utf8]{inputenc}
\ifdefined\DeclareUnicodeCharacter
% support both utf8 and utf8x syntaxes
  \ifdefined\DeclareUnicodeCharacterAsOptional
    \def\sphinxDUC#1{\DeclareUnicodeCharacter{"#1}}
  \else
    \let\sphinxDUC\DeclareUnicodeCharacter
  \fi
  \sphinxDUC{00A0}{\nobreakspace}
  \sphinxDUC{2500}{\sphinxunichar{2500}}
  \sphinxDUC{2502}{\sphinxunichar{2502}}
  \sphinxDUC{2514}{\sphinxunichar{2514}}
  \sphinxDUC{251C}{\sphinxunichar{251C}}
  \sphinxDUC{2572}{\textbackslash}
\fi
\usepackage{cmap}
\usepackage[T1]{fontenc}
\usepackage{amsmath,amssymb,amstext}
\usepackage{babel}



\usepackage{tgtermes}
\usepackage{tgheros}
\renewcommand{\ttdefault}{txtt}



\usepackage[Bjarne]{fncychap}
\usepackage{sphinx}

\fvset{fontsize=auto}
\usepackage{geometry}


% Include hyperref last.
\usepackage{hyperref}
% Fix anchor placement for figures with captions.
\usepackage{hypcap}% it must be loaded after hyperref.
% Set up styles of URL: it should be placed after hyperref.
\urlstyle{same}

\addto\captionsenglish{\renewcommand{\contentsname}{Contents:}}

\usepackage{sphinxmessages}
\setcounter{tocdepth}{1}



\title{ThermoElectric Documentation}
\date{Mar 28, 2022}
\release{}
\author{ThermoElectric}
\newcommand{\sphinxlogo}{\vbox{}}
\renewcommand{\releasename}{}
\makeindex
\begin{document}

\pagestyle{empty}
\sphinxmaketitle
\pagestyle{plain}
\sphinxtableofcontents
\pagestyle{normal}
\phantomsection\label{\detokenize{index::doc}}


\sphinxAtStartPar
ThermoElectric is a computational framework that computes electron transport coefficients with unique features to design
the nanoscale morphology of thermoelectric (TE) materials to obtain electron scattering that will enhance performance
through electron energy filtering. The code uses the linear Boltzmann transport equation to compute the electrical
properties of electrical conductivity, Seebeck coefficient, electron thermal conductivity, etc., under relaxation time
approximation.


\chapter{Installation}
\label{\detokenize{index:installation}}
\begin{sphinxVerbatim}[commandchars=\\\{\}]
\PYG{n}{gh} \PYG{n}{repo} \PYG{n}{clone} \PYG{n}{ariahosseini}\PYG{o}{/}\PYG{n}{ThermoElectric}
\PYG{n}{cd} \PYG{n}{ThermoElectric}
\PYG{n}{sudo} \PYG{n}{python} \PYG{n}{setup}\PYG{o}{.}\PYG{n}{py} \PYG{n}{install}
\end{sphinxVerbatim}


\section{Getting Started}
\label{\detokenize{getting_started:getting-started}}\label{\detokenize{getting_started::doc}}
\sphinxAtStartPar
This page details how to get started with ThermoElectric.


\subsection{Intrinsic Properties}
\label{\detokenize{getting_started:intrinsic-properties}}
\sphinxAtStartPar
Once installed, you can use the package. This example shows how to model the electron transport coefficients in
nano\sphinxhyphen{}structured silicon.

\sphinxAtStartPar
The following block of code computes the energy range {[}eV{]}, temperature range {[}K{]}, and the electronic band gap {[}eV{]}:

\begin{sphinxVerbatim}[commandchars=\\\{\}]
\PYG{k+kn}{import} \PYG{n+nn}{ThermoElectric} \PYG{k}{as} \PYG{n+nn}{TE}
\PYG{k+kn}{import} \PYG{n+nn}{numpy} \PYG{k}{as} \PYG{n+nn}{np}
\PYG{k+kn}{from} \PYG{n+nn}{numpy}\PYG{n+nn}{.}\PYG{n+nn}{linalg} \PYG{k+kn}{import} \PYG{n}{norm}
\PYG{k+kn}{from} \PYG{n+nn}{scipy}\PYG{n+nn}{.}\PYG{n+nn}{interpolate} \PYG{k+kn}{import} \PYG{n}{PchipInterpolator} \PYG{k}{as} \PYG{n}{interpolator}

\PYG{n}{energy\PYGZus{}min} \PYG{o}{=} \PYG{l+m+mf}{0.0}  \PYG{c+c1}{\PYGZsh{} Minimum energy level [eV]}
\PYG{n}{energy\PYGZus{}max} \PYG{o}{=} \PYG{l+m+mi}{1}  \PYG{c+c1}{\PYGZsh{} Maximum energy level [eV]}
\PYG{n}{num\PYGZus{}enrg\PYGZus{}sample} \PYG{o}{=} \PYG{n}{num\PYGZus{}enrg\PYGZus{}sample} \PYG{o}{=} \PYG{l+m+mi}{4000}  \PYG{c+c1}{\PYGZsh{} Number of energy points}

\PYG{n}{tmpr\PYGZus{}min} \PYG{o}{=} \PYG{l+m+mi}{300}  \PYG{c+c1}{\PYGZsh{} Minimum temperature [K]}
\PYG{n}{tmpr\PYGZus{}max} \PYG{o}{=} \PYG{l+m+mi}{1300}  \PYG{c+c1}{\PYGZsh{} Maximum temperature [K]}
\PYG{n}{tmpr\PYGZus{}step} \PYG{o}{=} \PYG{l+m+mi}{50}  \PYG{c+c1}{\PYGZsh{} Number of temperature points}

\PYG{n}{engr} \PYG{o}{=} \PYG{n}{TE}\PYG{o}{.}\PYG{n}{energy\PYGZus{}range}\PYG{p}{(}\PYG{n}{energy\PYGZus{}min} \PYG{o}{=} \PYG{n}{energy\PYGZus{}min}\PYG{p}{,} \PYG{n}{energy\PYGZus{}max} \PYG{o}{=} \PYG{n}{energy\PYGZus{}max}\PYG{p}{,}
                       \PYG{n}{sample\PYGZus{}size} \PYG{o}{=} \PYG{n}{num\PYGZus{}enrg\PYGZus{}sample}\PYG{p}{)}
\PYG{n}{tmpr} \PYG{o}{=} \PYG{n}{TE}\PYG{o}{.}\PYG{n}{temperature}\PYG{p}{(}\PYG{n}{temp\PYGZus{}min} \PYG{o}{=} \PYG{n}{tmpr\PYGZus{}min}\PYG{p}{,} \PYG{n}{temp\PYGZus{}max} \PYG{o}{=} \PYG{n}{tmpr\PYGZus{}max}\PYG{p}{,} \PYG{n}{del\PYGZus{}temp} \PYG{o}{=} \PYG{n}{tmpr\PYGZus{}step}\PYG{p}{)}
\PYG{n}{electronic\PYGZus{}gap} \PYG{o}{=} \PYG{n}{TE}\PYG{o}{.}\PYG{n}{band\PYGZus{}gap}\PYG{p}{(}\PYG{n}{Eg\PYGZus{}o} \PYG{o}{=} \PYG{l+m+mf}{1.17}\PYG{p}{,} \PYG{n}{Ao} \PYG{o}{=} \PYG{l+m+mf}{4.73e\PYGZhy{}4}\PYG{p}{,} \PYG{n}{Bo} \PYG{o}{=} \PYG{l+m+mi}{636}\PYG{p}{,} \PYG{n}{temp} \PYG{o}{=} \PYG{n}{tmpr}\PYG{p}{)}
\end{sphinxVerbatim}

\sphinxAtStartPar
ThermoElectric uses the following form for the band gap:
\begin{equation*}
\begin{split}\mathrm{E_g(T) = E_g(0) - \frac{A_o T^2}{T+B_o}}\end{split}
\end{equation*}
\sphinxAtStartPar
For the silicon, \(\mathrm{E_g(T) = 1.17\ [eV], A_o = 4.73 \times 10^{-4}\ [eV/K], B_o = 636\ [K]}\), are used.
For more details, see “Properties of Advanced Semiconductor Materials” by Michael E. Levinshtein.

\sphinxAtStartPar
Next step is to compute total carrier concentration.

\begin{sphinxVerbatim}[commandchars=\\\{\}]
\PYG{n}{carrier\PYGZus{}con} \PYG{o}{=} \PYG{n}{TE}\PYG{o}{.}\PYG{n}{carrier\PYGZus{}concentration}\PYG{p}{(}\PYG{n}{path\PYGZus{}extrinsic\PYGZus{}carrier} \PYG{o}{=}
                                       \PYG{l+s+s1}{\PYGZsq{}}\PYG{l+s+s1}{Exp\PYGZus{}Data/experimental\PYGZhy{}carrier\PYGZhy{}concentration\PYGZhy{}5pct\PYGZhy{}direction\PYGZhy{}up.txt}\PYG{l+s+s1}{\PYGZsq{}}\PYG{p}{,}
                                       \PYG{n}{band\PYGZus{}gap} \PYG{o}{=} \PYG{n}{electronic\PYGZus{}gap}\PYG{p}{,} \PYG{n}{Ao} \PYG{o}{=} \PYG{l+m+mf}{5.3e21}\PYG{p}{,} \PYG{n}{Bo} \PYG{o}{=} \PYG{l+m+mf}{3.5e21}\PYG{p}{,} \PYG{n}{temp} \PYG{o}{=} \PYG{n}{tmpr}\PYG{p}{)}
\end{sphinxVerbatim}

\sphinxAtStartPar
The intrinsic carrier concentration is computed using \(\mathrm{N_i = \sqrt{N_c N_v} \exp(\frac{E_g}{2k_B T})}\),
where \(\mathrm{N_c = A_o T^{3/2}}\) and \(\mathrm{N_v = B_o T^{3/2}}\) are the effective densities of states
in the conduction and valence bands, respectively. For the silicon,
\(\mathrm{A_o = 5.3 \times 10^{21}\ [m^{-3}K^{-3/2}], B_o = 3.5 \times 10^{21}\ [m^{-3}K^{-3/2}]}\), are used from
“Properties of Advanced Semiconductor Materials” by Michael E. Levinshtein.

\sphinxAtStartPar
We need to define the reciprocal space basis. For Si, the basis is defined as:

\begin{sphinxVerbatim}[commandchars=\\\{\}]
\PYG{n}{lattice\PYGZus{}parameter} \PYG{o}{=} \PYG{l+m+mf}{5.40e\PYGZhy{}10}  \PYG{c+c1}{\PYGZsh{} Si lattice parameter in [m]}
\PYG{n}{lattice\PYGZus{}vec} \PYG{o}{=} \PYG{n}{np}\PYG{o}{.}\PYG{n}{array}\PYG{p}{(}\PYG{p}{[}\PYG{p}{[}\PYG{l+m+mi}{1}\PYG{p}{,}\PYG{l+m+mi}{1}\PYG{p}{,}\PYG{l+m+mi}{0}\PYG{p}{]}\PYG{p}{,}\PYG{p}{[}\PYG{l+m+mi}{0}\PYG{p}{,}\PYG{l+m+mi}{1}\PYG{p}{,}\PYG{l+m+mi}{1}\PYG{p}{]}\PYG{p}{,}\PYG{p}{[}\PYG{l+m+mi}{1}\PYG{p}{,}\PYG{l+m+mi}{0}\PYG{p}{,}\PYG{l+m+mi}{1}\PYG{p}{]}\PYG{p}{]}\PYG{p}{)}\PYG{o}{*}\PYG{n}{lattice\PYGZus{}parameter}\PYG{o}{/}\PYG{l+m+mi}{2} \PYG{c+c1}{\PYGZsh{} lattice vector in [1/m]}
\PYG{n}{a\PYGZus{}rp} \PYG{o}{=} \PYG{n}{np}\PYG{o}{.}\PYG{n}{cross}\PYG{p}{(}\PYG{n}{lattice\PYGZus{}vec}\PYG{p}{[}\PYG{l+m+mi}{1}\PYG{p}{]}\PYG{p}{,} \PYG{n}{lattice\PYGZus{}vec}\PYG{p}{[}\PYG{l+m+mi}{2}\PYG{p}{]}\PYG{p}{)}\PYG{o}{/} \PYG{n}{np}\PYG{o}{.}\PYG{n}{dot}\PYG{p}{(}\PYG{n}{lattice\PYGZus{}vec}\PYG{p}{[}\PYG{l+m+mi}{0}\PYG{p}{]}\PYG{p}{,} \PYG{n}{np}\PYG{o}{.}\PYG{n}{cross}\PYG{p}{(}\PYG{n}{lattice\PYGZus{}vec}\PYG{p}{[}\PYG{l+m+mi}{1}\PYG{p}{]}\PYG{p}{,} \PYG{n}{lattice\PYGZus{}vec}\PYG{p}{[}\PYG{l+m+mi}{2}\PYG{p}{]}\PYG{p}{)}\PYG{p}{)}
\PYG{n}{b\PYGZus{}rp} \PYG{o}{=} \PYG{n}{np}\PYG{o}{.}\PYG{n}{cross}\PYG{p}{(}\PYG{n}{lattice\PYGZus{}vec}\PYG{p}{[}\PYG{l+m+mi}{2}\PYG{p}{]}\PYG{p}{,} \PYG{n}{lattice\PYGZus{}vec}\PYG{p}{[}\PYG{l+m+mi}{0}\PYG{p}{]}\PYG{p}{)}\PYG{o}{/} \PYG{n}{np}\PYG{o}{.}\PYG{n}{dot}\PYG{p}{(}\PYG{n}{lattice\PYGZus{}vec}\PYG{p}{[}\PYG{l+m+mi}{1}\PYG{p}{]}\PYG{p}{,} \PYG{n}{np}\PYG{o}{.}\PYG{n}{cross}\PYG{p}{(}\PYG{n}{lattice\PYGZus{}vec}\PYG{p}{[}\PYG{l+m+mi}{2}\PYG{p}{]}\PYG{p}{,} \PYG{n}{lattice\PYGZus{}vec}\PYG{p}{[}\PYG{l+m+mi}{0}\PYG{p}{]}\PYG{p}{)}\PYG{p}{)}
\PYG{n}{c\PYGZus{}rp} \PYG{o}{=} \PYG{n}{np}\PYG{o}{.}\PYG{n}{cross}\PYG{p}{(}\PYG{n}{lattice\PYGZus{}vec}\PYG{p}{[}\PYG{l+m+mi}{0}\PYG{p}{]}\PYG{p}{,} \PYG{n}{lattice\PYGZus{}vec}\PYG{p}{[}\PYG{l+m+mi}{1}\PYG{p}{]}\PYG{p}{)}\PYG{o}{/} \PYG{n}{np}\PYG{o}{.}\PYG{n}{dot}\PYG{p}{(}\PYG{n}{lattice\PYGZus{}vec}\PYG{p}{[}\PYG{l+m+mi}{2}\PYG{p}{]}\PYG{p}{,} \PYG{n}{np}\PYG{o}{.}\PYG{n}{cross}\PYG{p}{(}\PYG{n}{lattice\PYGZus{}vec}\PYG{p}{[}\PYG{l+m+mi}{0}\PYG{p}{]}\PYG{p}{,} \PYG{n}{lattice\PYGZus{}vec}\PYG{p}{[}\PYG{l+m+mi}{1}\PYG{p}{]}\PYG{p}{)}\PYG{p}{)}
\PYG{n}{recip\PYGZus{}lattice\PYGZus{}vec} \PYG{o}{=} \PYG{n}{np}\PYG{o}{.}\PYG{n}{array}\PYG{p}{(}\PYG{p}{[}\PYG{n}{a\PYGZus{}rp}\PYG{p}{,} \PYG{n}{b\PYGZus{}rp}\PYG{p}{,} \PYG{n}{c\PYGZus{}rp}\PYG{p}{]}\PYG{p}{)}  \PYG{c+c1}{\PYGZsh{} Reciprocal lattice vectors}
\end{sphinxVerbatim}

\sphinxAtStartPar
Next, we compute the band structure {[}eV{]}, group velocity {[}m/s{]}, and the density of states (1/m$^{\text{3}}$)

\begin{sphinxVerbatim}[commandchars=\\\{\}]
\PYG{n}{num\PYGZus{}kpoints} \PYG{o}{=} \PYG{l+m+mi}{800}  \PYG{c+c1}{\PYGZsh{} Number of kpoints in EIGENVAL}
\PYG{n}{num\PYGZus{}dos} \PYG{o}{=} \PYG{l+m+mi}{2000} \PYG{c+c1}{\PYGZsh{} Number of points in DOSCAR}
\PYG{n}{num\PYGZus{}bands} \PYG{o}{=} \PYG{l+m+mi}{8}  \PYG{c+c1}{\PYGZsh{} Number of bands in Si}
\PYG{n}{num\PYGZus{}qpoints} \PYG{o}{=} \PYG{l+m+mi}{200}  \PYG{c+c1}{\PYGZsh{} Number of q\PYGZhy{}points in desired band}
\PYG{n}{valley\PYGZus{}idx} \PYG{o}{=} \PYG{l+m+mi}{1118}  \PYG{c+c1}{\PYGZsh{} The index of valley in DOSCAR}
\PYG{n}{unitcell\PYGZus{}vol} \PYG{o}{=} \PYG{l+m+mi}{2}\PYG{o}{*}\PYG{l+m+mf}{19.70272e\PYGZhy{}30} \PYG{c+c1}{\PYGZsh{} Silicon unitcell volume}

\PYG{n}{dispersion} \PYG{o}{=} \PYG{n}{TE}\PYG{o}{.}\PYG{n}{band\PYGZus{}structure}\PYG{p}{(}\PYG{n}{path\PYGZus{}eigen} \PYG{o}{=} \PYG{l+s+s1}{\PYGZsq{}}\PYG{l+s+s1}{DFT\PYGZus{}Data/EIGENVAL}\PYG{l+s+s1}{\PYGZsq{}}\PYG{p}{,} \PYG{n}{skip\PYGZus{}lines} \PYG{o}{=} \PYG{l+m+mi}{6}\PYG{p}{,} \PYG{n}{num\PYGZus{}bands} \PYG{o}{=} \PYG{n}{num\PYGZus{}bands}\PYG{p}{,}
                               \PYG{n}{num\PYGZus{}kpoints} \PYG{o}{=} \PYG{n}{num\PYGZus{}kpoints}\PYG{p}{)}
\PYG{n}{kp} \PYG{o}{=} \PYG{n}{dispersion}\PYG{p}{[}\PYG{l+s+s1}{\PYGZsq{}}\PYG{l+s+s1}{k\PYGZus{}points}\PYG{l+s+s1}{\PYGZsq{}}\PYG{p}{]}
\PYG{n}{band\PYGZus{}struc} \PYG{o}{=} \PYG{n}{dispersion}\PYG{p}{[}\PYG{l+s+s1}{\PYGZsq{}}\PYG{l+s+s1}{electron\PYGZus{}dispersion}\PYG{l+s+s1}{\PYGZsq{}}\PYG{p}{]}

\PYG{n}{band\PYGZus{}dir} \PYG{o}{=} \PYG{n}{band\PYGZus{}str}\PYG{p}{[}\PYG{l+m+mi}{400}\PYG{p}{:} \PYG{l+m+mi}{400} \PYG{o}{+} \PYG{n}{num\PYGZus{}qpoints}\PYG{p}{,} \PYG{l+m+mi}{4}\PYG{p}{]}  \PYG{c+c1}{\PYGZsh{} The forth column is the conduction band}
\PYG{n}{min\PYGZus{}band} \PYG{o}{=} \PYG{n}{np}\PYG{o}{.}\PYG{n}{argmin}\PYG{p}{(}\PYG{n}{band\PYGZus{}dir}\PYG{p}{,} \PYG{n}{axis}\PYG{o}{=}\PYG{l+m+mi}{0}\PYG{p}{)}  \PYG{c+c1}{\PYGZsh{} The index of the conduction band valley}
\PYG{n}{max\PYGZus{}band} \PYG{o}{=} \PYG{n}{np}\PYG{o}{.}\PYG{n}{argmax}\PYG{p}{(}\PYG{n}{band\PYGZus{}dir}\PYG{p}{,} \PYG{n}{axis}\PYG{o}{=}\PYG{l+m+mi}{0}\PYG{p}{)}  \PYG{c+c1}{\PYGZsh{} The index of the maximum energy level in the conduction band}

\PYG{n}{kp\PYGZus{}rl} \PYG{o}{=} \PYG{l+m+mi}{2} \PYG{o}{*} \PYG{n}{np}\PYG{o}{.}\PYG{n}{pi} \PYG{o}{*} \PYG{n}{kp} \PYG{o}{@} \PYG{n}{RLv}  \PYG{c+c1}{\PYGZsh{} Wave\PYGZhy{}vectors in the reciprocal space}
\PYG{n}{kp\PYGZus{}mag} \PYG{o}{=} \PYG{n}{norm}\PYG{p}{(}\PYG{n}{kp\PYGZus{}rl}\PYG{p}{,} \PYG{n}{axis}\PYG{o}{=}\PYG{l+m+mi}{1}\PYG{p}{)}  \PYG{c+c1}{\PYGZsh{} The magnitude of the wave\PYGZhy{}vectors}
\PYG{n}{kp\PYGZus{}engr} \PYG{o}{=} \PYG{n}{kp\PYGZus{}mag}\PYG{p}{[}\PYG{l+m+mi}{400}\PYG{o}{+}\PYG{n}{max\PYGZus{}band}\PYG{p}{:} \PYG{l+m+mi}{400}\PYG{o}{+}\PYG{n}{min\PYGZus{}band}\PYG{p}{]}

\PYG{n}{energy\PYGZus{}kp} \PYG{o}{=} \PYG{n}{band\PYGZus{}struc}\PYG{p}{[}\PYG{l+m+mi}{400}\PYG{o}{+}\PYG{n}{max\PYGZus{}band}\PYG{p}{:} \PYG{l+m+mi}{400}\PYG{o}{+}\PYG{n}{min\PYGZus{}band}\PYG{p}{,} \PYG{l+m+mi}{4}\PYG{p}{]} \PYG{o}{\PYGZhy{}} \PYG{n}{band\PYGZus{}str}\PYG{p}{[}\PYG{l+m+mi}{400}\PYG{o}{+}\PYG{n}{min\PYGZus{}band}\PYG{p}{,} \PYG{l+m+mi}{4}\PYG{p}{]}
\PYG{n}{sort\PYGZus{}enrg} \PYG{o}{=} \PYG{n}{np}\PYG{o}{.}\PYG{n}{argsort}\PYG{p}{(}\PYG{n}{energy\PYGZus{}kp}\PYG{p}{,} \PYG{n}{axis}\PYG{o}{=}\PYG{l+m+mi}{0}\PYG{p}{)}
\PYG{c+c1}{\PYGZsh{} The electron group velocity}
\PYG{n}{grp\PYGZus{}velocity} \PYG{o}{=} \PYG{n}{TE}\PYG{o}{.}\PYG{n}{group\PYGZus{}velocity}\PYG{p}{(}\PYG{n}{kpoints} \PYG{o}{=} \PYG{n}{kp\PYGZus{}engr}\PYG{p}{[}\PYG{n}{sort\PYGZus{}enrg}\PYG{p}{]}\PYG{p}{,} \PYG{n}{energy\PYGZus{}kp} \PYG{o}{=} \PYG{n}{energy\PYGZus{}kp}\PYG{p}{[}\PYG{n}{sort\PYGZus{}enrg}\PYG{p}{]}\PYG{p}{,} \PYG{n}{energy} \PYG{o}{=} \PYG{n}{engr}\PYG{p}{)}
\PYG{c+c1}{\PYGZsh{} The electronic density of states}
\PYG{n}{e\PYGZus{}density} \PYG{o}{=} \PYG{n}{TE}\PYG{o}{.}\PYG{n}{electron\PYGZus{}density}\PYG{p}{(}\PYG{n}{path\PYGZus{}density} \PYG{o}{=} \PYG{l+s+s1}{\PYGZsq{}}\PYG{l+s+s1}{DFT\PYGZus{}Data/DOSCAR}\PYG{l+s+s1}{\PYGZsq{}}\PYG{p}{,} \PYG{n}{header\PYGZus{}lines} \PYG{o}{=} \PYG{l+m+mi}{6}\PYG{p}{,} \PYG{n}{unitcell\PYGZus{}volume}\PYG{o}{=} \PYG{n}{unitcell\PYGZus{}vol}\PYG{p}{,}
                                \PYG{n}{num\PYGZus{}dos\PYGZus{}points} \PYG{o}{=} \PYG{n}{num\PYGZus{}dos}\PYG{p}{,} \PYG{n}{valley\PYGZus{}point} \PYG{o}{=} \PYG{n}{valley\PYGZus{}idx}\PYG{p}{,} \PYG{n}{energy} \PYG{o}{=} \PYG{n}{engr}\PYG{p}{)}
\end{sphinxVerbatim}


\subsection{Fermi Energy Level}
\label{\detokenize{getting_started:fermi-energy-level}}
\sphinxAtStartPar
We can estimate the Fermi energy level using Joyce Dixon approximation

\begin{sphinxVerbatim}[commandchars=\\\{\}]
\PYG{n}{joyce\PYGZus{}dixon} \PYG{o}{=} \PYG{n}{TE}\PYG{o}{.}\PYG{n}{fermi\PYGZus{}level}\PYG{p}{(}\PYG{n}{carrier} \PYG{o}{=} \PYG{n}{carrier\PYGZus{}con}\PYG{p}{,} \PYG{n}{energy} \PYG{o}{=} \PYG{n}{engr}\PYG{p}{,} \PYG{n}{density} \PYG{o}{=} \PYG{n}{e\PYGZus{}density}\PYG{p}{,} \PYG{n}{Nc} \PYG{o}{=} \PYG{k+kc}{None}\PYG{p}{,}
                             \PYG{n}{Ao} \PYG{o}{=} \PYG{l+m+mf}{5.3e21}\PYG{p}{,} \PYG{n}{temp} \PYG{o}{=} \PYG{n}{tmpr}\PYG{p}{)}
\end{sphinxVerbatim}

\sphinxAtStartPar
Joyce Dixon approximate the Fermi level using
\(\mathrm{E_f = \ln\left(\frac{N_i}{Nc}\right) + \frac{1}{\sqrt{8}} \left(\frac{N_i}{Nc}\right) - (\frac{3}{16} - \frac{\sqrt{3}}{9}) \left(\frac{N_i}{Nc}\right)^2}\).

\sphinxAtStartPar
Next, we are using a self\sphinxhyphen{}consistent algorithm to accurately compute the Fermi level.

\begin{sphinxVerbatim}[commandchars=\\\{\}]
\PYG{n}{fermi} \PYG{o}{=} \PYG{n}{TE}\PYG{o}{.}\PYG{n}{fermi\PYGZus{}self\PYGZus{}consistent}\PYG{p}{(}\PYG{n}{carrier} \PYG{o}{=} \PYG{n}{carrier\PYGZus{}con}\PYG{p}{,} \PYG{n}{temp} \PYG{o}{=} \PYG{n}{tmpr}\PYG{p}{,} \PYG{n}{energy}\PYG{o}{=} \PYG{n}{engr}\PYG{p}{,} \PYG{n}{density}\PYG{o}{=} \PYG{n}{e\PYGZus{}density}\PYG{p}{,}
                                 \PYG{n}{fermi\PYGZus{}levels} \PYG{o}{=} \PYG{n}{joyce\PYGZus{}dixon}\PYG{p}{)}
\end{sphinxVerbatim}

\sphinxAtStartPar
Fermi distribution and its derivative (Fermi window) are computed as

\begin{sphinxVerbatim}[commandchars=\\\{\}]
\PYG{n}{k\PYGZus{}bolt} \PYG{o}{=} \PYG{l+m+mf}{8.617330350e\PYGZhy{}5}  \PYG{c+c1}{\PYGZsh{} Boltzmann constant in [eV/K]}
\PYG{n}{fermi\PYGZus{}dist} \PYG{o}{=} \PYG{n}{TE}\PYG{o}{.}\PYG{n}{fermi\PYGZus{}distribution}\PYG{p}{(}\PYG{n}{energy} \PYG{o}{=} \PYG{n}{engr}\PYG{p}{,} \PYG{n}{fermi\PYGZus{}level} \PYG{o}{=} \PYG{n}{fermi}\PYG{p}{[}\PYG{l+m+mi}{1}\PYG{p}{]}\PYG{p}{[}\PYG{n}{np}\PYG{o}{.}\PYG{n}{newaxis}\PYG{p}{,} \PYG{p}{:}\PYG{p}{]}\PYG{p}{,} \PYG{n}{temp} \PYG{o}{=} \PYG{n}{tmpr}\PYG{p}{)}
\PYG{n}{np}\PYG{o}{.}\PYG{n}{savetxt}\PYG{p}{(}\PYG{l+s+s2}{\PYGZdq{}}\PYG{l+s+s2}{Matlab\PYGZus{}Files/Ef.out}\PYG{l+s+s2}{\PYGZdq{}}\PYG{p}{,} \PYG{n}{fermi}\PYG{p}{[}\PYG{l+m+mi}{1}\PYG{p}{]} \PYG{o}{/} \PYG{n}{tmpr} \PYG{o}{/} \PYG{n}{k\PYGZus{}bolt}\PYG{p}{)}
\end{sphinxVerbatim}

\sphinxAtStartPar
We need Ef.out to compute the \sphinxhyphen{}0.5\sphinxhyphen{}order and 0.5\sphinxhyphen{}order Fermi\sphinxhyphen{}Dirac integral. The fermi.m is an script writen
by Natarajan and Mohankumar that may be used to evaluate the half\sphinxhyphen{}order Fermi\sphinxhyphen{}Dirac integral integrals. An alternative
python tool is dfint

\begin{sphinxVerbatim}[commandchars=\\\{\}]
\PYG{n}{pip} \PYG{n}{install} \PYG{n}{fdint}
\end{sphinxVerbatim}

\sphinxAtStartPar
The generalized Debye length is computed as \(L_D = \frac{e^2 N_c}{4 \pi \epsilon \epsilon_o k_B T }\left[F_{-1/2}(\eta) + \frac{15\alpha k_B T}{4}F_{1/2}(\eta)\right]\)

\begin{sphinxVerbatim}[commandchars=\\\{\}]
\PYG{n}{eps\PYGZus{}o} \PYG{o}{=} \PYG{l+m+mf}{8.854187817e\PYGZhy{}12}  \PYG{c+c1}{\PYGZsh{} Permittivity in vacuum F/m}
\PYG{n}{mass\PYGZus{}e} \PYG{o}{=} \PYG{l+m+mf}{9.109e\PYGZhy{}31}  \PYG{c+c1}{\PYGZsh{} Electron rest mass in Kg}
\PYG{n}{h\PYGZus{}bar} \PYG{o}{=} \PYG{l+m+mf}{6.582119e\PYGZhy{}16}  \PYG{c+c1}{\PYGZsh{} Reduced Planck constant in eV.s}
\PYG{n}{e2C} \PYG{o}{=} \PYG{l+m+mf}{1.6021765e\PYGZhy{}19}  \PYG{c+c1}{\PYGZsh{} e to Coulomb unit change}
\PYG{n}{nonparabolic\PYGZus{}term} \PYG{o}{=} \PYG{l+m+mf}{0.5}  \PYG{c+c1}{\PYGZsh{} Non\PYGZhy{}parabolic term}
\PYG{n}{dielectric} \PYG{o}{=} \PYG{l+m+mf}{11.7}  \PYG{c+c1}{\PYGZsh{} Relative dielectricity}

\PYG{n}{mass\PYGZus{}cond} \PYG{o}{=} \PYG{l+m+mf}{0.23} \PYG{o}{*} \PYG{n}{mass\PYGZus{}e} \PYG{o}{*} \PYG{p}{(}\PYG{l+m+mi}{1} \PYG{o}{+} \PYG{l+m+mi}{5} \PYG{o}{*} \PYG{n}{nonparabolic\PYGZus{}term} \PYG{o}{*} \PYG{n}{k\PYGZus{}bolt} \PYG{o}{*} \PYG{n}{tmpr}\PYG{p}{)}  \PYG{c+c1}{\PYGZsh{} Conduction band effective mass}
\PYG{n}{Nc} \PYG{o}{=} \PYG{l+m+mi}{2}\PYG{o}{*}\PYG{p}{(}\PYG{n}{mass\PYGZus{}cond} \PYG{o}{*} \PYG{n}{k\PYGZus{}bolt} \PYG{o}{*} \PYG{n}{tmpr} \PYG{o}{/} \PYG{n}{h\PYGZus{}bar}\PYG{o}{*}\PYG{o}{*}\PYG{l+m+mi}{2} \PYG{o}{/} \PYG{l+m+mi}{2}\PYG{o}{/} \PYG{n}{np}\PYG{o}{.}\PYG{n}{pi}\PYG{o}{/} \PYG{n}{e2C}\PYG{p}{)}\PYG{o}{*}\PYG{o}{*}\PYG{p}{(}\PYG{l+m+mf}{3.}\PYG{o}{/}\PYG{l+m+mi}{2}\PYG{p}{)}
\PYG{n}{fermi\PYGZus{}ints} \PYG{o}{=} \PYG{n}{np}\PYG{o}{.}\PYG{n}{loadtxt}\PYG{p}{(}\PYG{l+s+s2}{\PYGZdq{}}\PYG{l+s+s2}{Matlab\PYGZus{}Files/fermi\PYGZhy{}5pct\PYGZhy{}dir\PYGZhy{}up.out}\PYG{l+s+s2}{\PYGZdq{}}\PYG{p}{,} \PYG{n}{delimiter}\PYG{o}{=}\PYG{l+s+s1}{\PYGZsq{}}\PYG{l+s+s1}{,}\PYG{l+s+s1}{\PYGZsq{}}\PYG{p}{)}
\PYG{n}{screen\PYGZus{}len} \PYG{o}{=} \PYG{n}{np}\PYG{o}{.}\PYG{n}{sqrt}\PYG{p}{(}\PYG{l+m+mi}{1} \PYG{o}{/} \PYG{p}{(}\PYG{n}{Nc} \PYG{o}{/} \PYG{n}{dielectric} \PYG{o}{/} \PYG{n}{eps\PYGZus{}o} \PYG{o}{/} \PYG{n}{k\PYGZus{}bolt} \PYG{o}{/} \PYG{n}{tmpr} \PYG{o}{*} \PYG{n}{e2C} \PYG{o}{*}
                    \PYG{p}{(}\PYG{n}{fermi\PYGZus{}ints}\PYG{p}{[}\PYG{l+m+mi}{1}\PYG{p}{]} \PYG{o}{+} \PYG{l+m+mi}{15} \PYG{o}{*} \PYG{n}{nonparabolic\PYGZus{}term} \PYG{o}{*} \PYG{n}{k\PYGZus{}bolt} \PYG{o}{*} \PYG{n}{tmpr} \PYG{o}{/} \PYG{l+m+mi}{4} \PYG{o}{*} \PYG{n}{fermi\PYGZus{}ints}\PYG{p}{[}\PYG{l+m+mi}{0}\PYG{p}{]}\PYG{p}{)}\PYG{p}{)}\PYG{p}{)}
\end{sphinxVerbatim}


\subsection{Electron Lifetime}
\label{\detokenize{getting_started:electron-lifetime}}
\sphinxAtStartPar
The electron\sphinxhyphen{}phonon and electron\sphinxhyphen{}impurity scattering rates are computed as

\begin{sphinxVerbatim}[commandchars=\\\{\}]
\PYG{n}{bulk\PYGZus{}module} \PYG{o}{=} \PYG{l+m+mi}{98}  \PYG{c+c1}{\PYGZsh{} Bulk module (GPA)}
\PYG{n}{rho} \PYG{o}{=} \PYG{l+m+mi}{2329}  \PYG{c+c1}{\PYGZsh{} Mass density (Kg/m3)}
\PYG{n}{speed\PYGZus{}sound} \PYG{o}{=} \PYG{n}{np}\PYG{o}{.}\PYG{n}{sqrt}\PYG{p}{(}\PYG{n}{bulk\PYGZus{}module}\PYG{o}{/}\PYG{n}{rho}\PYG{p}{)}  \PYG{c+c1}{\PYGZsh{} Speed of sound}
\PYG{n}{num\PYGZus{}vally} \PYG{o}{=} \PYG{l+m+mi}{6}

\PYG{n}{tau\PYGZus{}ph} \PYG{o}{=} \PYG{n}{TE}\PYG{o}{.}\PYG{n}{tau\PYGZus{}p}\PYG{p}{(}\PYG{n}{energy} \PYG{o}{=} \PYG{n}{engr}\PYG{p}{,} \PYG{n}{alpha\PYGZus{}term} \PYG{o}{=} \PYG{n}{nonparabolic\PYGZus{}term} \PYG{p}{,} \PYG{n}{D\PYGZus{}v} \PYG{o}{=} \PYG{l+m+mf}{2.94}\PYG{p}{,} \PYG{n}{D\PYGZus{}a} \PYG{o}{=} \PYG{l+m+mf}{9.5}\PYG{p}{,} \PYG{n}{temp} \PYG{o}{=} \PYG{n}{tmpr}\PYG{p}{,}
                  \PYG{n}{vel\PYGZus{}sound} \PYG{o}{=} \PYG{n}{speed\PYGZus{}sound}\PYG{p}{,} \PYG{n}{DoS} \PYG{o}{=} \PYG{n}{e\PYGZus{}density}\PYG{p}{,} \PYG{n}{rho} \PYG{o}{=} \PYG{n}{rho}\PYG{p}{)}
\PYG{n}{tau\PYGZus{}imp} \PYG{o}{=} \PYG{n}{TE}\PYG{o}{.}\PYG{n}{tau\PYGZus{}strongly\PYGZus{}screened\PYGZus{}coulomb}\PYG{p}{(}\PYG{n}{DoS} \PYG{o}{=} \PYG{n}{e\PYGZus{}density}\PYG{p}{,} \PYG{n}{screen\PYGZus{}len} \PYG{o}{=} \PYG{n}{screen\PYGZus{}len}\PYG{p}{,} \PYG{n}{n\PYGZus{}imp} \PYG{o}{=} \PYG{n}{carrier\PYGZus{}con}\PYG{p}{,}
                                                \PYG{n}{dielectric} \PYG{o}{=} \PYG{n}{dielectric}\PYG{p}{)}
\PYG{n}{tau} \PYG{o}{=} \PYG{n}{TE}\PYG{o}{.}\PYG{n}{matthiessen}\PYG{p}{(}\PYG{n}{engr}\PYG{p}{,} \PYG{n}{num\PYGZus{}vally} \PYG{o}{*} \PYG{n}{tau\PYGZus{}ph}\PYG{p}{[}\PYG{l+s+s1}{\PYGZsq{}}\PYG{l+s+s1}{nonparabolic\PYGZus{}ph\PYGZus{}lifetime}\PYG{l+s+s1}{\PYGZsq{}}\PYG{p}{]}\PYG{p}{,} \PYG{n}{num\PYGZus{}vally} \PYG{o}{*} \PYG{n}{tau\PYGZus{}imp}\PYG{p}{)}
\end{sphinxVerbatim}

\sphinxAtStartPar
The following equations are used
\begin{equation*}
\begin{split}\mathrm{\tau_p(E)=\frac{\rho \nu_s^2 \hbar}{\pi \Phi_A^2 k_B T D(E)} \left ( \left[1-\frac{\alpha E}{1+2\alpha E}
        \left(1-\frac{\Phi_v}{\Phi_A} \right)\right]^2-\frac{8}{3} \frac{\alpha E(1+ \alpha E)}{(1+2 \alpha E)^2}
        \frac{D_v}{D_A} \right)^{-1}}\end{split}
\end{equation*}\begin{equation*}
\begin{split}\mathrm{\tau_{ion}(E)=\frac{\hbar}{\pi N_i \left(\frac{e^2 L_D^2}{4\pi \epsilon \epsilon_o}\right)^2 D(E)}}\end{split}
\end{equation*}
\sphinxAtStartPar
The Matthiessen’s rule is used to add the scattering rates.


\subsection{Electrical Properties}
\label{\detokenize{getting_started:electrical-properties}}
\sphinxAtStartPar
Finally the electrical properties are computed using

\begin{sphinxVerbatim}[commandchars=\\\{\}]
\PYG{n}{elec\PYGZus{}propt} \PYG{o}{=} \PYG{n}{TE}\PYG{o}{.}\PYG{n}{electrical\PYGZus{}properties}\PYG{p}{(}\PYG{n}{E} \PYG{o}{=} \PYG{n}{engr}\PYG{p}{,} \PYG{n}{DoS} \PYG{o}{=} \PYG{n}{e\PYGZus{}density}\PYG{p}{,} \PYG{n}{vg} \PYG{o}{=} \PYG{n}{grp\PYGZus{}velocity}\PYG{p}{,} \PYG{n}{Ef} \PYG{o}{=} \PYG{n}{fermi}\PYG{p}{[}\PYG{l+m+mi}{1}\PYG{p}{]}\PYG{p}{[}\PYG{n}{np}\PYG{o}{.}\PYG{n}{newaxis}\PYG{p}{,} \PYG{p}{:}\PYG{p}{]}\PYG{p}{,}
                                      \PYG{n}{dfdE} \PYG{o}{=} \PYG{n}{fermi\PYGZus{}dist}\PYG{p}{[}\PYG{l+m+mi}{1}\PYG{p}{]}\PYG{p}{,} \PYG{n}{temp} \PYG{o}{=} \PYG{n}{tmpr}\PYG{p}{,} \PYG{n}{tau} \PYG{o}{=} \PYG{n}{tau}\PYG{p}{)}
\end{sphinxVerbatim}


\section{References}
\label{\detokenize{ref:references}}\label{\detokenize{ref::doc}}
\sphinxAtStartPar
If you use ThermoElectric, please consider citing:

\begin{sphinxVerbatim}[commandchars=\\\{\}]
\PYG{n+nd}{@article}\PYG{p}{\PYGZob{}}\PYG{n}{doi}\PYG{p}{:}\PYG{l+m+mf}{10.1021}\PYG{o}{/}\PYG{n}{acsaem}\PYG{l+m+mf}{.0}\PYG{n}{c02640}\PYG{p}{,}
    \PYG{n}{author} \PYG{o}{=} \PYG{p}{\PYGZob{}}\PYG{n}{Hosseini}\PYG{p}{,} \PYG{n}{S}\PYG{o}{.} \PYG{n}{Aria} \PYG{o+ow}{and} \PYG{n}{Romano}\PYG{p}{,} \PYG{n}{Giuseppe} \PYG{o+ow}{and} \PYG{n}{Greaney}\PYG{p}{,} \PYG{n}{P}\PYG{o}{.} \PYG{n}{Alex}\PYG{p}{\PYGZcb{}}\PYG{p}{,}
    \PYG{n}{title} \PYG{o}{=} \PYG{p}{\PYGZob{}}\PYG{n}{Mitigating} \PYG{n}{the} \PYG{n}{Effect} \PYG{n}{of} \PYG{n}{Nanoscale} \PYG{n}{Porosity} \PYG{n}{on} \PYG{n}{Thermoelectric} \PYG{n}{Power} \PYG{n}{Factor} \PYG{n}{of} \PYG{n}{Si}\PYG{p}{\PYGZcb{}}\PYG{p}{,}
    \PYG{n}{journal} \PYG{o}{=} \PYG{p}{\PYGZob{}}\PYG{n}{ACS} \PYG{n}{Applied} \PYG{n}{Energy} \PYG{n}{Materials}\PYG{p}{\PYGZcb{}}\PYG{p}{,}
    \PYG{n}{volume} \PYG{o}{=} \PYG{p}{\PYGZob{}}\PYG{l+m+mi}{4}\PYG{p}{\PYGZcb{}}\PYG{p}{,}
    \PYG{n}{number} \PYG{o}{=} \PYG{p}{\PYGZob{}}\PYG{l+m+mi}{2}\PYG{p}{\PYGZcb{}}\PYG{p}{,}
    \PYG{n}{pages} \PYG{o}{=} \PYG{p}{\PYGZob{}}\PYG{l+m+mi}{1915}\PYG{o}{\PYGZhy{}}\PYG{l+m+mi}{1923}\PYG{p}{\PYGZcb{}}\PYG{p}{,}
    \PYG{n}{year} \PYG{o}{=} \PYG{p}{\PYGZob{}}\PYG{l+m+mi}{2021}\PYG{p}{\PYGZcb{}}\PYG{p}{,}
    \PYG{n}{doi} \PYG{o}{=} \PYG{p}{\PYGZob{}}\PYG{l+m+mf}{10.1021}\PYG{o}{/}\PYG{n}{acsaem}\PYG{l+m+mf}{.0}\PYG{n}{c02640}\PYG{p}{\PYGZcb{}}\PYG{p}{,}
    \PYG{n}{url} \PYG{o}{=} \PYG{p}{\PYGZob{}}\PYG{n}{https}\PYG{p}{:}\PYG{o}{/}\PYG{o}{/}\PYG{n}{doi}\PYG{o}{.}\PYG{n}{org}\PYG{o}{/}\PYG{l+m+mf}{10.1021}\PYG{o}{/}\PYG{n}{acsaem}\PYG{l+m+mf}{.0}\PYG{n}{c02640}\PYG{p}{\PYGZcb{}}\PYG{p}{,}
    \PYG{n}{eprint} \PYG{o}{=} \PYG{p}{\PYGZob{}}\PYG{n}{https}\PYG{p}{:}\PYG{o}{/}\PYG{o}{/}\PYG{n}{doi}\PYG{o}{.}\PYG{n}{org}\PYG{o}{/}\PYG{l+m+mf}{10.1021}\PYG{o}{/}\PYG{n}{acsaem}\PYG{l+m+mf}{.0}\PYG{n}{c02640}\PYG{p}{\PYGZcb{}}
    \PYG{p}{\PYGZcb{}}
\end{sphinxVerbatim}

\begin{sphinxVerbatim}[commandchars=\\\{\}]
\PYG{n+nd}{@Article}\PYG{p}{\PYGZob{}}\PYG{n}{nano11102591}\PYG{p}{,}
    \PYG{n}{AUTHOR} \PYG{o}{=} \PYG{p}{\PYGZob{}}\PYG{n}{Hosseini}\PYG{p}{,} \PYG{n}{S}\PYG{o}{.} \PYG{n}{Aria} \PYG{o+ow}{and} \PYG{n}{Romano}\PYG{p}{,} \PYG{n}{Giuseppe} \PYG{o+ow}{and} \PYG{n}{Greaney}\PYG{p}{,} \PYG{n}{P}\PYG{o}{.} \PYG{n}{Alex}\PYG{p}{\PYGZcb{}}\PYG{p}{,}
    \PYG{n}{TITLE} \PYG{o}{=} \PYG{p}{\PYGZob{}}\PYG{n}{Enhanced} \PYG{n}{Thermoelectric} \PYG{n}{Performance} \PYG{n}{of} \PYG{n}{Polycrystalline} \PYG{n}{Si0}\PYG{l+m+mf}{.8}\PYG{n}{Ge0}\PYG{l+m+mf}{.2} \PYG{n}{Alloys} \PYG{n}{through} \PYG{n}{the} \PYG{n}{Addition} \PYG{n}{of} \PYG{n}{Nanoscale} \PYG{n}{Porosity}\PYG{p}{\PYGZcb{}}\PYG{p}{,}
    \PYG{n}{JOURNAL} \PYG{o}{=} \PYG{p}{\PYGZob{}}\PYG{n}{Nanomaterials}\PYG{p}{\PYGZcb{}}\PYG{p}{,}
    \PYG{n}{VOLUME} \PYG{o}{=} \PYG{p}{\PYGZob{}}\PYG{l+m+mi}{11}\PYG{p}{\PYGZcb{}}\PYG{p}{,}
    \PYG{n}{YEAR} \PYG{o}{=} \PYG{p}{\PYGZob{}}\PYG{l+m+mi}{2021}\PYG{p}{\PYGZcb{}}\PYG{p}{,}
    \PYG{n}{NUMBER} \PYG{o}{=} \PYG{p}{\PYGZob{}}\PYG{l+m+mi}{10}\PYG{p}{\PYGZcb{}}\PYG{p}{,}
    \PYG{n}{ARTICLE}\PYG{o}{\PYGZhy{}}\PYG{n}{NUMBER} \PYG{o}{=} \PYG{p}{\PYGZob{}}\PYG{l+m+mi}{2591}\PYG{p}{\PYGZcb{}}\PYG{p}{,}
    \PYG{n}{URL} \PYG{o}{=} \PYG{p}{\PYGZob{}}\PYG{n}{https}\PYG{p}{:}\PYG{o}{/}\PYG{o}{/}\PYG{n}{www}\PYG{o}{.}\PYG{n}{mdpi}\PYG{o}{.}\PYG{n}{com}\PYG{o}{/}\PYG{l+m+mi}{2079}\PYG{o}{\PYGZhy{}}\PYG{l+m+mi}{4991}\PYG{o}{/}\PYG{l+m+mi}{11}\PYG{o}{/}\PYG{l+m+mi}{10}\PYG{o}{/}\PYG{l+m+mi}{2591}\PYG{p}{\PYGZcb{}}\PYG{p}{,}
    \PYG{n}{PubMedID} \PYG{o}{=} \PYG{p}{\PYGZob{}}\PYG{l+m+mi}{34685032}\PYG{p}{\PYGZcb{}}\PYG{p}{,}
    \PYG{n}{ISSN} \PYG{o}{=} \PYG{p}{\PYGZob{}}\PYG{l+m+mi}{2079}\PYG{o}{\PYGZhy{}}\PYG{l+m+mi}{4991}\PYG{p}{\PYGZcb{}}\PYG{p}{,}
    \PYG{n}{DOI} \PYG{o}{=} \PYG{p}{\PYGZob{}}\PYG{l+m+mf}{10.3390}\PYG{o}{/}\PYG{n}{nano11102591}\PYG{p}{\PYGZcb{}}
    \PYG{p}{\PYGZcb{}}
\end{sphinxVerbatim}

\begin{sphinxVerbatim}[commandchars=\\\{\}]
@article\PYGZob{}hosseini2021enhanced,
    title=\PYGZob{}Enhanced Thermoelectric ZT in the Tails of the Fermi Distribution via Electron Filtering by Nanoinclusions — Model Electron Transport in Nanocomposites\PYGZcb{},
    author=\PYGZob{}Hosseini, S Aria and Coleman, Devin and Bux, Sabah and Greaney, P Alex and Mangolini, Lorenzo\PYGZcb{},
    journal=\PYGZob{}arXiv preprint arXiv:2110.13375\PYGZcb{},
    year=\PYGZob{}2021\PYGZcb{}
    \PYGZcb{}
\end{sphinxVerbatim}


\section{API Documentation}
\label{\detokenize{api:api-documentation}}\label{\detokenize{api::doc}}

\begin{savenotes}\sphinxatlongtablestart\begin{longtable}[c]{\X{1}{2}\X{1}{2}}
\hline

\endfirsthead

\multicolumn{2}{c}%
{\makebox[0pt]{\sphinxtablecontinued{\tablename\ \thetable{} \textendash{} continued from previous page}}}\\
\hline

\endhead

\hline
\multicolumn{2}{r}{\makebox[0pt][r]{\sphinxtablecontinued{continues on next page}}}\\
\endfoot

\endlastfoot

\sphinxAtStartPar
{\hyperref[\detokenize{autosummary/ThermoElectric.energy_range:ThermoElectric.energy_range}]{\sphinxcrossref{\sphinxcode{\sphinxupquote{ThermoElectric.energy\_range}}}}}(energy\_min, ...)
&
\sphinxAtStartPar
Create a 2D array of energy space sampling
\\
\hline
\sphinxAtStartPar
{\hyperref[\detokenize{autosummary/ThermoElectric.temperature:ThermoElectric.temperature}]{\sphinxcrossref{\sphinxcode{\sphinxupquote{ThermoElectric.temperature}}}}}({[}temp\_min, ...{]})
&
\sphinxAtStartPar
Create a 2D array of temperature sampling
\\
\hline
\sphinxAtStartPar
{\hyperref[\detokenize{autosummary/ThermoElectric.fermi_distribution:ThermoElectric.fermi_distribution}]{\sphinxcrossref{\sphinxcode{\sphinxupquote{ThermoElectric.fermi\_distribution}}}}}(energy, ...)
&
\sphinxAtStartPar
A function to compute the Fermi distribution and the Fermi window — the first energy derivative of the Fermi level
\\
\hline
\sphinxAtStartPar
{\hyperref[\detokenize{autosummary/ThermoElectric.matthiessen:ThermoElectric.matthiessen}]{\sphinxcrossref{\sphinxcode{\sphinxupquote{ThermoElectric.matthiessen}}}}}(energy, *args)
&
\sphinxAtStartPar
A function to compute the total lifetime using Matthiessen\textquotesingle{}s rule
\\
\hline
\sphinxAtStartPar
{\hyperref[\detokenize{autosummary/ThermoElectric.kpoints:ThermoElectric.kpoints}]{\sphinxcrossref{\sphinxcode{\sphinxupquote{ThermoElectric.kpoints}}}}}(path2kpoints{[}, ...{]})
&
\sphinxAtStartPar
Create a 2D array of temperature sampling
\\
\hline
\sphinxAtStartPar
{\hyperref[\detokenize{autosummary/ThermoElectric.carrier_concentration:ThermoElectric.carrier_concentration}]{\sphinxcrossref{\sphinxcode{\sphinxupquote{ThermoElectric.carrier\_concentration}}}}}(...{[}, ...{]})
&
\sphinxAtStartPar
This function computes the carrier concentration.
\\
\hline
\sphinxAtStartPar
{\hyperref[\detokenize{autosummary/ThermoElectric.band_structure:ThermoElectric.band_structure}]{\sphinxcrossref{\sphinxcode{\sphinxupquote{ThermoElectric.band\_structure}}}}}(path\_eigen, ...)
&
\sphinxAtStartPar
A function to read "EIGENVAL" file
\\
\hline
\sphinxAtStartPar
{\hyperref[\detokenize{autosummary/ThermoElectric.electron_density:ThermoElectric.electron_density}]{\sphinxcrossref{\sphinxcode{\sphinxupquote{ThermoElectric.electron\_density}}}}}(...)
&
\sphinxAtStartPar
A function to read "DOSCAR" file
\\
\hline
\sphinxAtStartPar
{\hyperref[\detokenize{autosummary/ThermoElectric.band_gap:ThermoElectric.band_gap}]{\sphinxcrossref{\sphinxcode{\sphinxupquote{ThermoElectric.band\_gap}}}}}(Eg\_o, Ao, Bo{[}, temp{]})
&
\sphinxAtStartPar
This method uses Eg(T)=Eg(T=0)\sphinxhyphen{}Ao*T**2/(T+Bo) to approximate the temperature dependency of the dielectrics band gap.
\\
\hline
\sphinxAtStartPar
{\hyperref[\detokenize{autosummary/ThermoElectric.analytical_dos:ThermoElectric.analytical_dos}]{\sphinxcrossref{\sphinxcode{\sphinxupquote{ThermoElectric.analytical\_dos}}}}}(range\_energy, ...)
&
\sphinxAtStartPar
This function approximate the electron density of state for parabolic and non\sphinxhyphen{}parabolic bands
\\
\hline
\sphinxAtStartPar
{\hyperref[\detokenize{autosummary/ThermoElectric.fermi_level:ThermoElectric.fermi_level}]{\sphinxcrossref{\sphinxcode{\sphinxupquote{ThermoElectric.fermi\_level}}}}}(carrier, energy, ...)
&
\sphinxAtStartPar
This function uses Joice Dixon approximation to predict Ef and thereby the carrier concentration at each temperature A good reference book is "Principles of Semiconductor Devices" by Sima Dimitrijev.
\\
\hline
\sphinxAtStartPar
{\hyperref[\detokenize{autosummary/ThermoElectric.fermi_self_consistent:ThermoElectric.fermi_self_consistent}]{\sphinxcrossref{\sphinxcode{\sphinxupquote{ThermoElectric.fermi\_self\_consistent}}}}}(...)
&
\sphinxAtStartPar
Self\sphinxhyphen{}consistent calculation of the Fermi level from a given carrier concentration using Joyce Dixon approximation as the initial guess for degenerate semiconductors.
\\
\hline
\sphinxAtStartPar
{\hyperref[\detokenize{autosummary/ThermoElectric.group_velocity:ThermoElectric.group_velocity}]{\sphinxcrossref{\sphinxcode{\sphinxupquote{ThermoElectric.group\_velocity}}}}}(kpoints, ...)
&
\sphinxAtStartPar
Group velocity is the derivation of band structure from DFT.
\\
\hline
\sphinxAtStartPar
{\hyperref[\detokenize{autosummary/ThermoElectric.analytical_group_velocity:ThermoElectric.analytical_group_velocity}]{\sphinxcrossref{\sphinxcode{\sphinxupquote{ThermoElectric.analytical\_group\_velocity}}}}}(...)
&
\sphinxAtStartPar
This method approximate the group velocity near the conduction band edge in case no DFT calculation is available.
\\
\hline
\sphinxAtStartPar
{\hyperref[\detokenize{autosummary/ThermoElectric.tau_p:ThermoElectric.tau_p}]{\sphinxcrossref{\sphinxcode{\sphinxupquote{ThermoElectric.tau\_p}}}}}(energy, alpha\_term, ...)
&
\sphinxAtStartPar
Electron\sphinxhyphen{}phonon scattering rate using Ravich model
\\
\hline
\sphinxAtStartPar
{\hyperref[\detokenize{autosummary/ThermoElectric.tau_strongly_screened_coulomb:ThermoElectric.tau_strongly_screened_coulomb}]{\sphinxcrossref{\sphinxcode{\sphinxupquote{ThermoElectric.tau\_strongly\_screened\_coulomb}}}}}(...)
&
\sphinxAtStartPar
Electron\sphinxhyphen{}impurity scattering model in highly doped dielectrics
\\
\hline
\sphinxAtStartPar
{\hyperref[\detokenize{autosummary/ThermoElectric.tau_screened_coulomb:ThermoElectric.tau_screened_coulomb}]{\sphinxcrossref{\sphinxcode{\sphinxupquote{ThermoElectric.tau\_screened\_coulomb}}}}}(energy, ...)
&
\sphinxAtStartPar
Electron\sphinxhyphen{}ion scattering rate — Brook\sphinxhyphen{}Herring model
\\
\hline
\sphinxAtStartPar
{\hyperref[\detokenize{autosummary/ThermoElectric.tau_unscreened_coulomb:ThermoElectric.tau_unscreened_coulomb}]{\sphinxcrossref{\sphinxcode{\sphinxupquote{ThermoElectric.tau\_unscreened\_coulomb}}}}}(...)
&
\sphinxAtStartPar
Electron\sphinxhyphen{}ion scattering rate for shallow dopants \textasciitilde{}10\textasciicircum{}18 1/cm\textasciicircum{}3 (no screening effect is considered)
\\
\hline
\sphinxAtStartPar
{\hyperref[\detokenize{autosummary/ThermoElectric.tau_2d_cylinder:ThermoElectric.tau_2d_cylinder}]{\sphinxcrossref{\sphinxcode{\sphinxupquote{ThermoElectric.tau\_2d\_cylinder}}}}}(energy, ...)
&
\sphinxAtStartPar
A fast algorithm that uses Fermi’s golden rule to compute the energy dependent electron scattering rate from cylindrical nano\sphinxhyphen{}particles or nano\sphinxhyphen{}scale pores infinitely extended perpendicular to the current.
\\
\hline
\sphinxAtStartPar
{\hyperref[\detokenize{autosummary/ThermoElectric.tau3D_spherical:ThermoElectric.tau3D_spherical}]{\sphinxcrossref{\sphinxcode{\sphinxupquote{ThermoElectric.tau3D\_spherical}}}}}(num\_kpoints, ...)
&
\sphinxAtStartPar
A fast algorithm that uses Fermi’s golden rule to compute the energy dependent electron scattering rate from spherical nano\sphinxhyphen{}particles or nano\sphinxhyphen{}scale pores.
\\
\hline
\sphinxAtStartPar
{\hyperref[\detokenize{autosummary/ThermoElectric.electrical_properties:ThermoElectric.electrical_properties}]{\sphinxcrossref{\sphinxcode{\sphinxupquote{ThermoElectric.electrical\_properties}}}}}(E, DoS, ...)
&
\sphinxAtStartPar
This function returns electronic properties.
\\
\hline
\sphinxAtStartPar
{\hyperref[\detokenize{autosummary/ThermoElectric.filtering_effect:ThermoElectric.filtering_effect}]{\sphinxcrossref{\sphinxcode{\sphinxupquote{ThermoElectric.filtering\_effect}}}}}(Uo, E, DoS, ...)
&
\sphinxAtStartPar
This function returns electric properties for the ideal filtering —  where all the electrons up to a cutoff energy level of Uo are completely hindered.
\\
\hline
\sphinxAtStartPar
{\hyperref[\detokenize{autosummary/ThermoElectric.phenomenological:ThermoElectric.phenomenological}]{\sphinxcrossref{\sphinxcode{\sphinxupquote{ThermoElectric.phenomenological}}}}}(Uo, tau\_o, ...)
&
\sphinxAtStartPar
This function returns electric properties for the phenomenological filtering —  where a frequency independent lifetime of tau\_o is imposed to all the electrons up to a cutoff energy level of Uo
\\
\hline
\end{longtable}\sphinxatlongtableend\end{savenotes}


\subsection{ThermoElectric.energy\_range}
\label{\detokenize{autosummary/ThermoElectric.energy_range:thermoelectric-energy-range}}\label{\detokenize{autosummary/ThermoElectric.energy_range::doc}}\index{energy\_range() (in module ThermoElectric)@\spxentry{energy\_range()}\spxextra{in module ThermoElectric}}

\begin{fulllineitems}
\phantomsection\label{\detokenize{autosummary/ThermoElectric.energy_range:ThermoElectric.energy_range}}\pysiglinewithargsret{\sphinxcode{\sphinxupquote{ThermoElectric.}}\sphinxbfcode{\sphinxupquote{energy\_range}}}{\emph{\DUrole{n}{energy\_min}\DUrole{p}{:}\DUrole{w}{  }\DUrole{n}{float}}, \emph{\DUrole{n}{energy\_max}\DUrole{p}{:}\DUrole{w}{  }\DUrole{n}{float}}, \emph{\DUrole{n}{sample\_size}\DUrole{p}{:}\DUrole{w}{  }\DUrole{n}{int}}}{{ $\rightarrow$ numpy.ndarray}}
\sphinxAtStartPar
Create a 2D array of energy space sampling
\begin{quote}\begin{description}
\item[{Parameters}] \leavevmode\begin{itemize}
\item {} 
\sphinxAtStartPar
\sphinxstylestrong{energy\_min} (\sphinxstyleemphasis{float}) \textendash{} minimum energy level of electron in conduction band

\item {} 
\sphinxAtStartPar
\sphinxstylestrong{energy\_max} (\sphinxstyleemphasis{float}) \textendash{} maximum energy level of electron in conduction band

\item {} 
\sphinxAtStartPar
\sphinxstylestrong{sample\_size} (\sphinxstyleemphasis{int}) \textendash{} Sampling size of the energy space

\end{itemize}

\item[{Returns}] \leavevmode
\sphinxAtStartPar
\sphinxstylestrong{energy\_sample} \textendash{} Energy sampling with the size of {[}1, sample\_size{]}

\item[{Return type}] \leavevmode
\sphinxAtStartPar
np.ndarray

\end{description}\end{quote}

\end{fulllineitems}



\subsection{ThermoElectric.temperature}
\label{\detokenize{autosummary/ThermoElectric.temperature:thermoelectric-temperature}}\label{\detokenize{autosummary/ThermoElectric.temperature::doc}}\index{temperature() (in module ThermoElectric)@\spxentry{temperature()}\spxextra{in module ThermoElectric}}

\begin{fulllineitems}
\phantomsection\label{\detokenize{autosummary/ThermoElectric.temperature:ThermoElectric.temperature}}\pysiglinewithargsret{\sphinxcode{\sphinxupquote{ThermoElectric.}}\sphinxbfcode{\sphinxupquote{temperature}}}{\emph{\DUrole{n}{temp\_min}\DUrole{p}{:}\DUrole{w}{  }\DUrole{n}{float}\DUrole{w}{  }\DUrole{o}{=}\DUrole{w}{  }\DUrole{default_value}{300}}, \emph{\DUrole{n}{temp\_max}\DUrole{p}{:}\DUrole{w}{  }\DUrole{n}{float}\DUrole{w}{  }\DUrole{o}{=}\DUrole{w}{  }\DUrole{default_value}{1301}}, \emph{\DUrole{n}{del\_temp}\DUrole{p}{:}\DUrole{w}{  }\DUrole{n}{float}\DUrole{w}{  }\DUrole{o}{=}\DUrole{w}{  }\DUrole{default_value}{100}}}{{ $\rightarrow$ numpy.ndarray}}
\sphinxAtStartPar
Create a 2D array of temperature sampling
\begin{quote}\begin{description}
\item[{Parameters}] \leavevmode\begin{itemize}
\item {} 
\sphinxAtStartPar
\sphinxstylestrong{temp\_min} (\sphinxstyleemphasis{float}) \textendash{} minimum temperature

\item {} 
\sphinxAtStartPar
\sphinxstylestrong{temp\_max} (\sphinxstyleemphasis{float}) \textendash{} maximum temperature

\item {} 
\sphinxAtStartPar
\sphinxstylestrong{del\_temp} (\sphinxstyleemphasis{float}) \textendash{} Sampling size of the temperature range

\end{itemize}

\item[{Returns}] \leavevmode
\sphinxAtStartPar
\sphinxstylestrong{temp\_range} \textendash{} Temperature sampling with the size of {[}1, del\_temp{]}

\item[{Return type}] \leavevmode
\sphinxAtStartPar
np.ndarray

\end{description}\end{quote}

\end{fulllineitems}



\subsection{ThermoElectric.fermi\_distribution}
\label{\detokenize{autosummary/ThermoElectric.fermi_distribution:thermoelectric-fermi-distribution}}\label{\detokenize{autosummary/ThermoElectric.fermi_distribution::doc}}\index{fermi\_distribution() (in module ThermoElectric)@\spxentry{fermi\_distribution()}\spxextra{in module ThermoElectric}}

\begin{fulllineitems}
\phantomsection\label{\detokenize{autosummary/ThermoElectric.fermi_distribution:ThermoElectric.fermi_distribution}}\pysiglinewithargsret{\sphinxcode{\sphinxupquote{ThermoElectric.}}\sphinxbfcode{\sphinxupquote{fermi\_distribution}}}{\emph{\DUrole{n}{energy}\DUrole{p}{:}\DUrole{w}{  }\DUrole{n}{numpy.ndarray}}, \emph{\DUrole{n}{fermi\_level}\DUrole{p}{:}\DUrole{w}{  }\DUrole{n}{numpy.ndarray}}, \emph{\DUrole{n}{temp}\DUrole{p}{:}\DUrole{w}{  }\DUrole{n}{Optional\DUrole{p}{{[}}numpy.ndarray\DUrole{p}{{]}}}\DUrole{w}{  }\DUrole{o}{=}\DUrole{w}{  }\DUrole{default_value}{None}}}{{ $\rightarrow$ numpy.ndarray}}
\sphinxAtStartPar
A function to compute the Fermi distribution and
the Fermi window — the first energy derivative of the Fermi level
\begin{quote}\begin{description}
\item[{Parameters}] \leavevmode\begin{itemize}
\item {} 
\sphinxAtStartPar
\sphinxstylestrong{energy} (\sphinxstyleemphasis{np.ndarray}) \textendash{} Energy range in conduction band

\item {} 
\sphinxAtStartPar
\sphinxstylestrong{fermi\_level} (\sphinxstyleemphasis{np.ndarray}) \textendash{} Fermi level

\item {} 
\sphinxAtStartPar
\sphinxstylestrong{temp} (\sphinxstyleemphasis{np.ndarray}) \textendash{} Temperature range

\end{itemize}

\item[{Returns}] \leavevmode
\sphinxAtStartPar
\sphinxstylestrong{fermi} \textendash{} The first row is the Fermi distribution and the second row is the derivative (Fermi window)

\item[{Return type}] \leavevmode
\sphinxAtStartPar
np.ndarray

\end{description}\end{quote}

\end{fulllineitems}



\subsection{ThermoElectric.matthiessen}
\label{\detokenize{autosummary/ThermoElectric.matthiessen:thermoelectric-matthiessen}}\label{\detokenize{autosummary/ThermoElectric.matthiessen::doc}}\index{matthiessen() (in module ThermoElectric)@\spxentry{matthiessen()}\spxextra{in module ThermoElectric}}

\begin{fulllineitems}
\phantomsection\label{\detokenize{autosummary/ThermoElectric.matthiessen:ThermoElectric.matthiessen}}\pysiglinewithargsret{\sphinxcode{\sphinxupquote{ThermoElectric.}}\sphinxbfcode{\sphinxupquote{matthiessen}}}{\emph{\DUrole{n}{energy}\DUrole{p}{:}\DUrole{w}{  }\DUrole{n}{numpy.ndarray}}, \emph{\DUrole{o}{*}\DUrole{n}{args}}}{{ $\rightarrow$ numpy.ndarray}}
\sphinxAtStartPar
A function to compute the total lifetime using Matthiessen’s rule
\begin{quote}\begin{description}
\item[{Parameters}] \leavevmode\begin{itemize}
\item {} 
\sphinxAtStartPar
\sphinxstylestrong{energy} (\sphinxstyleemphasis{np.ndarray}) \textendash{} Energy levels

\item {} 
\sphinxAtStartPar
\sphinxstylestrong{*args} (\sphinxstyleemphasis{np.ndarray}) \textendash{} Electron lifetime from different scattering sources

\end{itemize}

\item[{Returns}] \leavevmode
\sphinxAtStartPar
\sphinxstylestrong{tau} \textendash{} The total electron lifetime

\item[{Return type}] \leavevmode
\sphinxAtStartPar
np.ndarray

\end{description}\end{quote}

\end{fulllineitems}



\subsection{ThermoElectric.kpoints}
\label{\detokenize{autosummary/ThermoElectric.kpoints:thermoelectric-kpoints}}\label{\detokenize{autosummary/ThermoElectric.kpoints::doc}}\index{kpoints() (in module ThermoElectric)@\spxentry{kpoints()}\spxextra{in module ThermoElectric}}

\begin{fulllineitems}
\phantomsection\label{\detokenize{autosummary/ThermoElectric.kpoints:ThermoElectric.kpoints}}\pysiglinewithargsret{\sphinxcode{\sphinxupquote{ThermoElectric.}}\sphinxbfcode{\sphinxupquote{kpoints}}}{\emph{\DUrole{n}{path2kpoints}\DUrole{p}{:}\DUrole{w}{  }\DUrole{n}{str}}, \emph{\DUrole{n}{delimiter}\DUrole{p}{:}\DUrole{w}{  }\DUrole{n}{Optional\DUrole{p}{{[}}str\DUrole{p}{{]}}}\DUrole{w}{  }\DUrole{o}{=}\DUrole{w}{  }\DUrole{default_value}{None}}, \emph{\DUrole{n}{skip\_rows}\DUrole{p}{:}\DUrole{w}{  }\DUrole{n}{int}\DUrole{w}{  }\DUrole{o}{=}\DUrole{w}{  }\DUrole{default_value}{0}}}{{ $\rightarrow$ numpy.ndarray}}
\sphinxAtStartPar
Create a 2D array of temperature sampling
\begin{quote}\begin{description}
\item[{Parameters}] \leavevmode\begin{itemize}
\item {} 
\sphinxAtStartPar
\sphinxstylestrong{path2kpoints} (\sphinxstyleemphasis{str}) \textendash{} Path to kpoints file

\item {} 
\sphinxAtStartPar
\sphinxstylestrong{delimiter} (\sphinxstyleemphasis{str}) \textendash{} Default it None for ,

\item {} 
\sphinxAtStartPar
\sphinxstylestrong{skip\_rows} (\sphinxstyleemphasis{int}) \textendash{} Number of lines to skip, default is 0

\end{itemize}

\item[{Returns}] \leavevmode
\sphinxAtStartPar
\sphinxstylestrong{wave\_points} \textendash{} Wave vectors

\item[{Return type}] \leavevmode
\sphinxAtStartPar
np.ndarray

\end{description}\end{quote}

\end{fulllineitems}



\subsection{ThermoElectric.carrier\_concentration}
\label{\detokenize{autosummary/ThermoElectric.carrier_concentration:thermoelectric-carrier-concentration}}\label{\detokenize{autosummary/ThermoElectric.carrier_concentration::doc}}\index{carrier\_concentration() (in module ThermoElectric)@\spxentry{carrier\_concentration()}\spxextra{in module ThermoElectric}}

\begin{fulllineitems}
\phantomsection\label{\detokenize{autosummary/ThermoElectric.carrier_concentration:ThermoElectric.carrier_concentration}}\pysiglinewithargsret{\sphinxcode{\sphinxupquote{ThermoElectric.}}\sphinxbfcode{\sphinxupquote{carrier\_concentration}}}{\emph{\DUrole{n}{path\_extrinsic\_carrier}\DUrole{p}{:}\DUrole{w}{  }\DUrole{n}{str}}, \emph{\DUrole{n}{band\_gap}\DUrole{p}{:}\DUrole{w}{  }\DUrole{n}{numpy.ndarray}}, \emph{\DUrole{n}{Ao}\DUrole{p}{:}\DUrole{w}{  }\DUrole{n}{Optional\DUrole{p}{{[}}float\DUrole{p}{{]}}}\DUrole{w}{  }\DUrole{o}{=}\DUrole{w}{  }\DUrole{default_value}{None}}, \emph{\DUrole{n}{Bo}\DUrole{p}{:}\DUrole{w}{  }\DUrole{n}{Optional\DUrole{p}{{[}}float\DUrole{p}{{]}}}\DUrole{w}{  }\DUrole{o}{=}\DUrole{w}{  }\DUrole{default_value}{None}}, \emph{\DUrole{n}{Nc}\DUrole{p}{:}\DUrole{w}{  }\DUrole{n}{Optional\DUrole{p}{{[}}float\DUrole{p}{{]}}}\DUrole{w}{  }\DUrole{o}{=}\DUrole{w}{  }\DUrole{default_value}{None}}, \emph{\DUrole{n}{Nv}\DUrole{p}{:}\DUrole{w}{  }\DUrole{n}{Optional\DUrole{p}{{[}}float\DUrole{p}{{]}}}\DUrole{w}{  }\DUrole{o}{=}\DUrole{w}{  }\DUrole{default_value}{None}}, \emph{\DUrole{n}{temp}\DUrole{p}{:}\DUrole{w}{  }\DUrole{n}{Optional\DUrole{p}{{[}}numpy.ndarray\DUrole{p}{{]}}}\DUrole{w}{  }\DUrole{o}{=}\DUrole{w}{  }\DUrole{default_value}{None}}}{{ $\rightarrow$ numpy.ndarray}}
\sphinxAtStartPar
This function computes the carrier concentration. The extrinsic carrier concentration is from experiments.
The following formula is used to compute intrinsic carrier concentration: n = sqrt(Nc*Nv)*exp(\sphinxhyphen{}Eg/kB/T/2)
A good reference book is “Principles of Semiconductor Devices” by Sima Dimitrijev
\begin{quote}\begin{description}
\item[{Parameters}] \leavevmode\begin{itemize}
\item {} 
\sphinxAtStartPar
\sphinxstylestrong{path\_extrinsic\_carrier} (\sphinxstyleemphasis{str}) \textendash{} Path to kpoints file

\item {} 
\sphinxAtStartPar
\sphinxstylestrong{band\_gap} (\sphinxstyleemphasis{np.ndarray}) \textendash{} The electronic band gap

\item {} 
\sphinxAtStartPar
\sphinxstylestrong{Ao} (\sphinxstyleemphasis{float}) \textendash{} Experimentally fitted parameter (Nc \textasciitilde{} Ao*T\textasciicircum{}(3/2))

\item {} 
\sphinxAtStartPar
\sphinxstylestrong{Bo} (\sphinxstyleemphasis{float}) \textendash{} Experimentally fitted parameter (Nv \textasciitilde{} Ao*T\textasciicircum{}(3/2))

\item {} 
\sphinxAtStartPar
\sphinxstylestrong{Nc} (\sphinxstyleemphasis{float}) \textendash{} The effective densities of states in the conduction band

\item {} 
\sphinxAtStartPar
\sphinxstylestrong{Nv} (\sphinxstyleemphasis{float}) \textendash{} The effective densities of states in the conduction band

\item {} 
\sphinxAtStartPar
\sphinxstylestrong{temp} (\sphinxstyleemphasis{np.ndarray}) \textendash{} Temperature range

\end{itemize}

\item[{Returns}] \leavevmode
\sphinxAtStartPar
\sphinxstylestrong{carrier} \textendash{} The total carrier concentration

\item[{Return type}] \leavevmode
\sphinxAtStartPar
np.ndarray

\end{description}\end{quote}

\end{fulllineitems}



\subsection{ThermoElectric.band\_structure}
\label{\detokenize{autosummary/ThermoElectric.band_structure:thermoelectric-band-structure}}\label{\detokenize{autosummary/ThermoElectric.band_structure::doc}}\index{band\_structure() (in module ThermoElectric)@\spxentry{band\_structure()}\spxextra{in module ThermoElectric}}

\begin{fulllineitems}
\phantomsection\label{\detokenize{autosummary/ThermoElectric.band_structure:ThermoElectric.band_structure}}\pysiglinewithargsret{\sphinxcode{\sphinxupquote{ThermoElectric.}}\sphinxbfcode{\sphinxupquote{band\_structure}}}{\emph{\DUrole{n}{path\_eigen}\DUrole{p}{:}\DUrole{w}{  }\DUrole{n}{str}}, \emph{\DUrole{n}{skip\_lines}\DUrole{p}{:}\DUrole{w}{  }\DUrole{n}{int}}, \emph{\DUrole{n}{num\_bands}\DUrole{p}{:}\DUrole{w}{  }\DUrole{n}{int}}, \emph{\DUrole{n}{num\_kpoints}\DUrole{p}{:}\DUrole{w}{  }\DUrole{n}{int}}}{{ $\rightarrow$ dict}}
\sphinxAtStartPar
A function to read “EIGENVAL” file
\begin{quote}\begin{description}
\item[{Parameters}] \leavevmode\begin{itemize}
\item {} 
\sphinxAtStartPar
\sphinxstylestrong{path\_eigen} (\sphinxstyleemphasis{str}) \textendash{} Path to EIGENVAL file

\item {} 
\sphinxAtStartPar
\sphinxstylestrong{skip\_lines} (\sphinxstyleemphasis{int}) \textendash{} Number of lines to skip

\item {} 
\sphinxAtStartPar
\sphinxstylestrong{num\_bands} (\sphinxstyleemphasis{int}) \textendash{} Number of bands

\item {} 
\sphinxAtStartPar
\sphinxstylestrong{num\_kpoints} (\sphinxstyleemphasis{int}) \textendash{} number of wave vectors

\end{itemize}

\item[{Returns}] \leavevmode
\sphinxAtStartPar
\sphinxstylestrong{dispersion} \textendash{} Band structure

\item[{Return type}] \leavevmode
\sphinxAtStartPar
dict

\end{description}\end{quote}

\end{fulllineitems}



\subsection{ThermoElectric.electron\_density}
\label{\detokenize{autosummary/ThermoElectric.electron_density:thermoelectric-electron-density}}\label{\detokenize{autosummary/ThermoElectric.electron_density::doc}}\index{electron\_density() (in module ThermoElectric)@\spxentry{electron\_density()}\spxextra{in module ThermoElectric}}

\begin{fulllineitems}
\phantomsection\label{\detokenize{autosummary/ThermoElectric.electron_density:ThermoElectric.electron_density}}\pysiglinewithargsret{\sphinxcode{\sphinxupquote{ThermoElectric.}}\sphinxbfcode{\sphinxupquote{electron\_density}}}{\emph{\DUrole{n}{path\_density}\DUrole{p}{:}\DUrole{w}{  }\DUrole{n}{str}}, \emph{\DUrole{n}{header\_lines}\DUrole{p}{:}\DUrole{w}{  }\DUrole{n}{int}}, \emph{\DUrole{n}{num\_dos\_points}\DUrole{p}{:}\DUrole{w}{  }\DUrole{n}{int}}, \emph{\DUrole{n}{unitcell\_volume}\DUrole{p}{:}\DUrole{w}{  }\DUrole{n}{float}}, \emph{\DUrole{n}{valley\_point}\DUrole{p}{:}\DUrole{w}{  }\DUrole{n}{int}}, \emph{\DUrole{n}{energy}\DUrole{p}{:}\DUrole{w}{  }\DUrole{n}{numpy.ndarray}}}{{ $\rightarrow$ numpy.ndarray}}
\sphinxAtStartPar
A function to read “DOSCAR” file
\begin{quote}\begin{description}
\item[{Parameters}] \leavevmode\begin{itemize}
\item {} 
\sphinxAtStartPar
\sphinxstylestrong{path\_density} (\sphinxstyleemphasis{str}) \textendash{} Path to DOSCAR file

\item {} 
\sphinxAtStartPar
\sphinxstylestrong{header\_lines} (\sphinxstyleemphasis{int}) \textendash{} Number of lines to skip

\item {} 
\sphinxAtStartPar
\sphinxstylestrong{num\_dos\_points} (\sphinxstyleemphasis{int}) \textendash{} Number of points in DOSCAR

\item {} 
\sphinxAtStartPar
\sphinxstylestrong{unitcell\_volume} (\sphinxstyleemphasis{float}) \textendash{} The unit cell volume is in {[}m{]}

\item {} 
\sphinxAtStartPar
\sphinxstylestrong{valley\_point} (\sphinxstyleemphasis{int}) \textendash{} Where valley is located in DOSCAR

\item {} 
\sphinxAtStartPar
\sphinxstylestrong{energy} (\sphinxstyleemphasis{np.ndarray}) \textendash{} The energy range

\end{itemize}

\item[{Returns}] \leavevmode
\sphinxAtStartPar
\sphinxstylestrong{density} \textendash{} Electron density of states

\item[{Return type}] \leavevmode
\sphinxAtStartPar
np.ndarray

\end{description}\end{quote}

\end{fulllineitems}



\subsection{ThermoElectric.band\_gap}
\label{\detokenize{autosummary/ThermoElectric.band_gap:thermoelectric-band-gap}}\label{\detokenize{autosummary/ThermoElectric.band_gap::doc}}\index{band\_gap() (in module ThermoElectric)@\spxentry{band\_gap()}\spxextra{in module ThermoElectric}}

\begin{fulllineitems}
\phantomsection\label{\detokenize{autosummary/ThermoElectric.band_gap:ThermoElectric.band_gap}}\pysiglinewithargsret{\sphinxcode{\sphinxupquote{ThermoElectric.}}\sphinxbfcode{\sphinxupquote{band\_gap}}}{\emph{\DUrole{n}{Eg\_o}\DUrole{p}{:}\DUrole{w}{  }\DUrole{n}{float}}, \emph{\DUrole{n}{Ao}\DUrole{p}{:}\DUrole{w}{  }\DUrole{n}{float}}, \emph{\DUrole{n}{Bo}\DUrole{p}{:}\DUrole{w}{  }\DUrole{n}{float}}, \emph{\DUrole{n}{temp}\DUrole{p}{:}\DUrole{w}{  }\DUrole{n}{Optional\DUrole{p}{{[}}numpy.ndarray\DUrole{p}{{]}}}\DUrole{w}{  }\DUrole{o}{=}\DUrole{w}{  }\DUrole{default_value}{None}}}{{ $\rightarrow$ numpy.ndarray}}
\sphinxAtStartPar
This method uses Eg(T)=Eg(T=0)\sphinxhyphen{}Ao*T**2/(T+Bo) to approximate the temperature dependency of the dielectrics band gap.
A good reference is “Properties of Advanced Semiconductor Materials” by Michael E. Levinshtein.
\begin{quote}\begin{description}
\item[{Parameters}] \leavevmode\begin{itemize}
\item {} 
\sphinxAtStartPar
\sphinxstylestrong{Eg\_o} (\sphinxstyleemphasis{float}) \textendash{} The band gap at zero Kelvin

\item {} 
\sphinxAtStartPar
\sphinxstylestrong{Ao} (\sphinxstyleemphasis{float}) \textendash{} Experimentally fitted parameter

\item {} 
\sphinxAtStartPar
\sphinxstylestrong{Bo} (\sphinxstyleemphasis{float}) \textendash{} Experimentally fitted parameter

\item {} 
\sphinxAtStartPar
\sphinxstylestrong{temp} (\sphinxstyleemphasis{np.ndarray}) \textendash{} Temperature range

\end{itemize}

\item[{Returns}] \leavevmode
\sphinxAtStartPar
\sphinxstylestrong{Eg} \textendash{} Temperature\sphinxhyphen{}dependent band gap

\item[{Return type}] \leavevmode
\sphinxAtStartPar
np.ndarray

\end{description}\end{quote}

\end{fulllineitems}



\subsection{ThermoElectric.analytical\_dos}
\label{\detokenize{autosummary/ThermoElectric.analytical_dos:thermoelectric-analytical-dos}}\label{\detokenize{autosummary/ThermoElectric.analytical_dos::doc}}\index{analytical\_dos() (in module ThermoElectric)@\spxentry{analytical\_dos()}\spxextra{in module ThermoElectric}}

\begin{fulllineitems}
\phantomsection\label{\detokenize{autosummary/ThermoElectric.analytical_dos:ThermoElectric.analytical_dos}}\pysiglinewithargsret{\sphinxcode{\sphinxupquote{ThermoElectric.}}\sphinxbfcode{\sphinxupquote{analytical\_dos}}}{\emph{\DUrole{n}{range\_energy}\DUrole{p}{:}\DUrole{w}{  }\DUrole{n}{numpy.ndarray}}, \emph{\DUrole{n}{electron\_eff\_mass}\DUrole{p}{:}\DUrole{w}{  }\DUrole{n}{float}}, \emph{\DUrole{n}{nonparabolic\_term}\DUrole{p}{:}\DUrole{w}{  }\DUrole{n}{numpy.ndarray}}}{{ $\rightarrow$ dict}}\begin{quote}

\sphinxAtStartPar
This function approximate the electron density of state for parabolic and non\sphinxhyphen{}parabolic bands
in case DFT calculation is not available.
\end{quote}
\begin{quote}\begin{description}
\item[{Parameters}] \leavevmode\begin{itemize}
\item {} 
\sphinxAtStartPar
\sphinxstylestrong{range\_energy} (\sphinxstyleemphasis{np.ndarray}) \textendash{} Electron energy range

\item {} 
\sphinxAtStartPar
\sphinxstylestrong{electron\_eff\_mass} (\sphinxstyleemphasis{float}) \textendash{} Electron effective mass

\item {} 
\sphinxAtStartPar
\sphinxstylestrong{nonparabolic\_term} (\sphinxstyleemphasis{np.ndarray}) \textendash{} Non\sphinxhyphen{}parabolic term (shows the mixture of S and P orbitals)

\end{itemize}

\item[{Returns}] \leavevmode
\sphinxAtStartPar
\sphinxstylestrong{DoS} \textendash{} The DoS{[}‘DoS\_nonparabolic’{]} is phonon density of states for non\sphinxhyphen{}parabolic band and
the DoS{[}‘DoS\_parabolic’{]} is the phonon density of states for non\sphinxhyphen{}parabolic.

\item[{Return type}] \leavevmode
\sphinxAtStartPar
dict

\end{description}\end{quote}

\end{fulllineitems}



\subsection{ThermoElectric.fermi\_level}
\label{\detokenize{autosummary/ThermoElectric.fermi_level:thermoelectric-fermi-level}}\label{\detokenize{autosummary/ThermoElectric.fermi_level::doc}}\index{fermi\_level() (in module ThermoElectric)@\spxentry{fermi\_level()}\spxextra{in module ThermoElectric}}

\begin{fulllineitems}
\phantomsection\label{\detokenize{autosummary/ThermoElectric.fermi_level:ThermoElectric.fermi_level}}\pysiglinewithargsret{\sphinxcode{\sphinxupquote{ThermoElectric.}}\sphinxbfcode{\sphinxupquote{fermi\_level}}}{\emph{\DUrole{n}{carrier}\DUrole{p}{:}\DUrole{w}{  }\DUrole{n}{numpy.ndarray}}, \emph{\DUrole{n}{energy}\DUrole{p}{:}\DUrole{w}{  }\DUrole{n}{numpy.ndarray}}, \emph{\DUrole{n}{density}\DUrole{p}{:}\DUrole{w}{  }\DUrole{n}{numpy.ndarray}}, \emph{\DUrole{n}{Nc}\DUrole{p}{:}\DUrole{w}{  }\DUrole{n}{Optional\DUrole{p}{{[}}float\DUrole{p}{{]}}}\DUrole{w}{  }\DUrole{o}{=}\DUrole{w}{  }\DUrole{default_value}{None}}, \emph{\DUrole{n}{Ao}\DUrole{p}{:}\DUrole{w}{  }\DUrole{n}{Optional\DUrole{p}{{[}}float\DUrole{p}{{]}}}\DUrole{w}{  }\DUrole{o}{=}\DUrole{w}{  }\DUrole{default_value}{None}}, \emph{\DUrole{n}{temp}\DUrole{p}{:}\DUrole{w}{  }\DUrole{n}{Optional\DUrole{p}{{[}}numpy.ndarray\DUrole{p}{{]}}}\DUrole{w}{  }\DUrole{o}{=}\DUrole{w}{  }\DUrole{default_value}{None}}}{{ $\rightarrow$ numpy.ndarray}}
\sphinxAtStartPar
This function uses Joice Dixon approximation to predict Ef and thereby the carrier concentration at each temperature
A good reference book is “Principles of Semiconductor Devices” by Sima Dimitrijev.
\begin{quote}\begin{description}
\item[{Parameters}] \leavevmode\begin{itemize}
\item {} 
\sphinxAtStartPar
\sphinxstylestrong{carrier} (\sphinxstyleemphasis{np.ndarray}) \textendash{} Total carrier concentration

\item {} 
\sphinxAtStartPar
\sphinxstylestrong{energy} (\sphinxstyleemphasis{np.ndarray}) \textendash{} The electron energy level

\item {} 
\sphinxAtStartPar
\sphinxstylestrong{density} (\sphinxstyleemphasis{np.ndarray}) \textendash{} The electron density of states

\item {} 
\sphinxAtStartPar
\sphinxstylestrong{Nc} (\sphinxstyleemphasis{float}) \textendash{} The effective densities of states in the conduction band

\item {} 
\sphinxAtStartPar
\sphinxstylestrong{Ao} (\sphinxstyleemphasis{float}) \textendash{} Experimentally fitted parameter (Nc \textasciitilde{} Ao*T\textasciicircum{}(3/2))

\item {} 
\sphinxAtStartPar
\sphinxstylestrong{temp} (\sphinxstyleemphasis{np.ndarray}) \textendash{} Temperature range

\end{itemize}

\item[{Returns}] \leavevmode
\sphinxAtStartPar
\sphinxstylestrong{output} \textendash{} The first row is the carrier concentration and the second one is the Fermi level

\item[{Return type}] \leavevmode
\sphinxAtStartPar
np.ndarray

\end{description}\end{quote}

\end{fulllineitems}



\subsection{ThermoElectric.fermi\_self\_consistent}
\label{\detokenize{autosummary/ThermoElectric.fermi_self_consistent:thermoelectric-fermi-self-consistent}}\label{\detokenize{autosummary/ThermoElectric.fermi_self_consistent::doc}}\index{fermi\_self\_consistent() (in module ThermoElectric)@\spxentry{fermi\_self\_consistent()}\spxextra{in module ThermoElectric}}

\begin{fulllineitems}
\phantomsection\label{\detokenize{autosummary/ThermoElectric.fermi_self_consistent:ThermoElectric.fermi_self_consistent}}\pysiglinewithargsret{\sphinxcode{\sphinxupquote{ThermoElectric.}}\sphinxbfcode{\sphinxupquote{fermi\_self\_consistent}}}{\emph{\DUrole{n}{carrier}\DUrole{p}{:}\DUrole{w}{  }\DUrole{n}{numpy.ndarray}}, \emph{\DUrole{n}{temp}\DUrole{p}{:}\DUrole{w}{  }\DUrole{n}{numpy.ndarray}}, \emph{\DUrole{n}{energy}\DUrole{p}{:}\DUrole{w}{  }\DUrole{n}{numpy.ndarray}}, \emph{\DUrole{n}{density}\DUrole{p}{:}\DUrole{w}{  }\DUrole{n}{numpy.ndarray}}, \emph{\DUrole{n}{fermi\_levels}\DUrole{p}{:}\DUrole{w}{  }\DUrole{n}{numpy.ndarray}}}{{ $\rightarrow$ numpy.ndarray}}
\sphinxAtStartPar
Self\sphinxhyphen{}consistent calculation of the Fermi level from a given carrier concentration
using Joyce Dixon approximation as the initial guess for degenerate semiconductors.
As a default value of 4000 sampling points in energy range from Ef(JD)\sphinxhyphen{}0.4 eV up to
Ef(JD)+0.2 is considered.This looks reasonable in most cases. The index is printed out
if it reaches the extreme index of (0) or (4000), increase energy range.
\begin{quote}\begin{description}
\item[{Parameters}] \leavevmode\begin{itemize}
\item {} 
\sphinxAtStartPar
\sphinxstylestrong{carrier} (\sphinxstyleemphasis{np.ndarray}) \textendash{} Total carrier concentration

\item {} 
\sphinxAtStartPar
\sphinxstylestrong{energy} (\sphinxstyleemphasis{np.ndarray}) \textendash{} The electron energy level

\item {} 
\sphinxAtStartPar
\sphinxstylestrong{density} (\sphinxstyleemphasis{np.ndarray}) \textendash{} The electron density of states

\item {} 
\sphinxAtStartPar
\sphinxstylestrong{fermi\_levels} (\sphinxstyleemphasis{np.ndarray}) \textendash{} Joyce Dixon femi level approximation as the initial guess

\item {} 
\sphinxAtStartPar
\sphinxstylestrong{temp} (\sphinxstyleemphasis{np.ndarray}) \textendash{} Temperature range

\end{itemize}

\item[{Returns}] \leavevmode
\sphinxAtStartPar
\sphinxstylestrong{output} \textendash{} The first row is the carrier concentration and the second one is the Fermi level

\item[{Return type}] \leavevmode
\sphinxAtStartPar
np.ndarray

\end{description}\end{quote}

\end{fulllineitems}



\subsection{ThermoElectric.group\_velocity}
\label{\detokenize{autosummary/ThermoElectric.group_velocity:thermoelectric-group-velocity}}\label{\detokenize{autosummary/ThermoElectric.group_velocity::doc}}\index{group\_velocity() (in module ThermoElectric)@\spxentry{group\_velocity()}\spxextra{in module ThermoElectric}}

\begin{fulllineitems}
\phantomsection\label{\detokenize{autosummary/ThermoElectric.group_velocity:ThermoElectric.group_velocity}}\pysiglinewithargsret{\sphinxcode{\sphinxupquote{ThermoElectric.}}\sphinxbfcode{\sphinxupquote{group\_velocity}}}{\emph{\DUrole{n}{kpoints}\DUrole{p}{:}\DUrole{w}{  }\DUrole{n}{numpy.ndarray}}, \emph{\DUrole{n}{energy\_kp}\DUrole{p}{:}\DUrole{w}{  }\DUrole{n}{numpy.ndarray}}, \emph{\DUrole{n}{energy}\DUrole{p}{:}\DUrole{w}{  }\DUrole{n}{numpy.ndarray}}}{{ $\rightarrow$ numpy.ndarray}}
\sphinxAtStartPar
Group velocity is the derivation of band structure from DFT. Linear BTE needs single band data.
Reciprocal lattice vector is required
\begin{quote}\begin{description}
\item[{Parameters}] \leavevmode\begin{itemize}
\item {} 
\sphinxAtStartPar
\sphinxstylestrong{kpoints} (\sphinxstyleemphasis{np.ndarray}) \textendash{} Wave vectors

\item {} 
\sphinxAtStartPar
\sphinxstylestrong{energy\_kp} (\sphinxstyleemphasis{np.ndarray}) \textendash{} Energy for each wave vector

\item {} 
\sphinxAtStartPar
\sphinxstylestrong{energy} (\sphinxstyleemphasis{np.ndarray}) \textendash{} Energy range

\end{itemize}

\item[{Returns}] \leavevmode
\sphinxAtStartPar
\sphinxstylestrong{velocity} \textendash{} Group velocity

\item[{Return type}] \leavevmode
\sphinxAtStartPar
np.ndarray

\end{description}\end{quote}

\end{fulllineitems}



\subsection{ThermoElectric.analytical\_group\_velocity}
\label{\detokenize{autosummary/ThermoElectric.analytical_group_velocity:thermoelectric-analytical-group-velocity}}\label{\detokenize{autosummary/ThermoElectric.analytical_group_velocity::doc}}\index{analytical\_group\_velocity() (in module ThermoElectric)@\spxentry{analytical\_group\_velocity()}\spxextra{in module ThermoElectric}}

\begin{fulllineitems}
\phantomsection\label{\detokenize{autosummary/ThermoElectric.analytical_group_velocity:ThermoElectric.analytical_group_velocity}}\pysiglinewithargsret{\sphinxcode{\sphinxupquote{ThermoElectric.}}\sphinxbfcode{\sphinxupquote{analytical\_group\_velocity}}}{\emph{\DUrole{n}{energy}\DUrole{p}{:}\DUrole{w}{  }\DUrole{n}{numpy.ndarray}}, \emph{\DUrole{n}{lattice\_parameter}\DUrole{p}{:}\DUrole{w}{  }\DUrole{n}{numpy.ndarray}}, \emph{\DUrole{n}{num\_kpoints}\DUrole{p}{:}\DUrole{w}{  }\DUrole{n}{numpy.ndarray}}, \emph{\DUrole{n}{m\_eff}\DUrole{p}{:}\DUrole{w}{  }\DUrole{n}{numpy.ndarray}}, \emph{\DUrole{n}{valley}\DUrole{p}{:}\DUrole{w}{  }\DUrole{n}{numpy.ndarray}}, \emph{\DUrole{n}{dk\_len}\DUrole{p}{:}\DUrole{w}{  }\DUrole{n}{numpy.ndarray}}, \emph{\DUrole{n}{alpha\_term}\DUrole{p}{:}\DUrole{w}{  }\DUrole{n}{numpy.ndarray}}}{{ $\rightarrow$ numpy.ndarray}}
\sphinxAtStartPar
This method approximate the group velocity near the conduction band edge
in case no DFT calculation is available.
\begin{quote}\begin{description}
\item[{Parameters}] \leavevmode\begin{itemize}
\item {} 
\sphinxAtStartPar
\sphinxstylestrong{energy} (\sphinxstyleemphasis{np.ndarray}) \textendash{} Energy range

\item {} 
\sphinxAtStartPar
\sphinxstylestrong{lattice\_parameter} (\sphinxstyleemphasis{np.ndarray}) \textendash{} Lattice parameter

\item {} 
\sphinxAtStartPar
\sphinxstylestrong{num\_kpoints} (\sphinxstyleemphasis{np.ndarray}) \textendash{} Number of wave vectors

\item {} 
\sphinxAtStartPar
\sphinxstylestrong{m\_eff} (\sphinxstyleemphasis{np.ndarray}) \textendash{} Effective mass along different axis

\item {} 
\sphinxAtStartPar
\sphinxstylestrong{valley} (\sphinxstyleemphasis{np.ndarray}) \textendash{} Conduction valley

\item {} 
\sphinxAtStartPar
\sphinxstylestrong{dk\_len} (\sphinxstyleemphasis{np.ndarray}) \textendash{} magnitude of wave vectors

\item {} 
\sphinxAtStartPar
\sphinxstylestrong{alpha\_term} (\sphinxstyleemphasis{np.ndarray}) \textendash{} Non\sphinxhyphen{}parabolic term

\end{itemize}

\item[{Returns}] \leavevmode
\sphinxAtStartPar
\sphinxstylestrong{velocity} \textendash{} Group velocity

\item[{Return type}] \leavevmode
\sphinxAtStartPar
np.ndarray

\end{description}\end{quote}

\end{fulllineitems}



\subsection{ThermoElectric.tau\_p}
\label{\detokenize{autosummary/ThermoElectric.tau_p:thermoelectric-tau-p}}\label{\detokenize{autosummary/ThermoElectric.tau_p::doc}}\index{tau\_p() (in module ThermoElectric)@\spxentry{tau\_p()}\spxextra{in module ThermoElectric}}

\begin{fulllineitems}
\phantomsection\label{\detokenize{autosummary/ThermoElectric.tau_p:ThermoElectric.tau_p}}\pysiglinewithargsret{\sphinxcode{\sphinxupquote{ThermoElectric.}}\sphinxbfcode{\sphinxupquote{tau\_p}}}{\emph{\DUrole{n}{energy}\DUrole{p}{:}\DUrole{w}{  }\DUrole{n}{numpy.ndarray}}, \emph{\DUrole{n}{alpha\_term}\DUrole{p}{:}\DUrole{w}{  }\DUrole{n}{numpy.ndarray}}, \emph{\DUrole{n}{D\_v}\DUrole{p}{:}\DUrole{w}{  }\DUrole{n}{float}}, \emph{\DUrole{n}{D\_a}\DUrole{p}{:}\DUrole{w}{  }\DUrole{n}{float}}, \emph{\DUrole{n}{temp}\DUrole{p}{:}\DUrole{w}{  }\DUrole{n}{numpy.ndarray}}, \emph{\DUrole{n}{vel\_sound}\DUrole{p}{:}\DUrole{w}{  }\DUrole{n}{float}}, \emph{\DUrole{n}{DoS}\DUrole{p}{:}\DUrole{w}{  }\DUrole{n}{numpy.ndarray}}, \emph{\DUrole{n}{rho}\DUrole{p}{:}\DUrole{w}{  }\DUrole{n}{float}}}{{ $\rightarrow$ dict}}
\sphinxAtStartPar
Electron\sphinxhyphen{}phonon scattering rate using Ravich model
\begin{quote}\begin{description}
\item[{Parameters}] \leavevmode\begin{itemize}
\item {} 
\sphinxAtStartPar
\sphinxstylestrong{energy} (\sphinxstyleemphasis{np.ndarray}) \textendash{} Energy range

\item {} 
\sphinxAtStartPar
\sphinxstylestrong{alpha\_term} (\sphinxstyleemphasis{np.ndarray}) \textendash{} Non\sphinxhyphen{}parabolic term

\item {} 
\sphinxAtStartPar
\sphinxstylestrong{D\_v} (\sphinxstyleemphasis{float}) \textendash{} Hole deformation potential

\item {} 
\sphinxAtStartPar
\sphinxstylestrong{D\_a} (\sphinxstyleemphasis{float}) \textendash{} Electron deformation potential

\item {} 
\sphinxAtStartPar
\sphinxstylestrong{temp} (\sphinxstyleemphasis{np.ndarray}) \textendash{} Temperature

\item {} 
\sphinxAtStartPar
\sphinxstylestrong{vel\_sound} (\sphinxstyleemphasis{float}) \textendash{} Sound velocity

\item {} 
\sphinxAtStartPar
\sphinxstylestrong{DoS} (\sphinxstyleemphasis{np.ndarray}) \textendash{} Density of state

\item {} 
\sphinxAtStartPar
\sphinxstylestrong{rho} (\sphinxstyleemphasis{float}) \textendash{} Mass density

\end{itemize}

\item[{Returns}] \leavevmode
\sphinxAtStartPar
\sphinxstylestrong{output} \textendash{} parabolic and non\sphinxhyphen{}parabolic electron\sphinxhyphen{}phonon lifetime

\item[{Return type}] \leavevmode
\sphinxAtStartPar
dict

\end{description}\end{quote}

\end{fulllineitems}



\subsection{ThermoElectric.tau\_strongly\_screened\_coulomb}
\label{\detokenize{autosummary/ThermoElectric.tau_strongly_screened_coulomb:thermoelectric-tau-strongly-screened-coulomb}}\label{\detokenize{autosummary/ThermoElectric.tau_strongly_screened_coulomb::doc}}\index{tau\_strongly\_screened\_coulomb() (in module ThermoElectric)@\spxentry{tau\_strongly\_screened\_coulomb()}\spxextra{in module ThermoElectric}}

\begin{fulllineitems}
\phantomsection\label{\detokenize{autosummary/ThermoElectric.tau_strongly_screened_coulomb:ThermoElectric.tau_strongly_screened_coulomb}}\pysiglinewithargsret{\sphinxcode{\sphinxupquote{ThermoElectric.}}\sphinxbfcode{\sphinxupquote{tau\_strongly\_screened\_coulomb}}}{\emph{\DUrole{n}{DoS}\DUrole{p}{:}\DUrole{w}{  }\DUrole{n}{numpy.ndarray}}, \emph{\DUrole{n}{screen\_len}\DUrole{p}{:}\DUrole{w}{  }\DUrole{n}{numpy.ndarray}}, \emph{\DUrole{n}{n\_imp}\DUrole{p}{:}\DUrole{w}{  }\DUrole{n}{numpy.ndarray}}, \emph{\DUrole{n}{dielectric}\DUrole{p}{:}\DUrole{w}{  }\DUrole{n}{float}}}{{ $\rightarrow$ numpy.ndarray}}
\sphinxAtStartPar
Electron\sphinxhyphen{}impurity scattering model in highly doped dielectrics

\sphinxAtStartPar
Note that for highly doped semiconductors, screen length plays a significant role,
therefor should be computed carefully. Highly suggest to use following matlab file “Fermi.m”
from: \sphinxurl{https://www.mathworks.com/matlabcentral/fileexchange/13616-fermi}

\sphinxAtStartPar
If committed to use python, the package “dfint” works with python2
pip install fdint
\begin{quote}\begin{description}
\item[{Parameters}] \leavevmode\begin{itemize}
\item {} 
\sphinxAtStartPar
\sphinxstylestrong{DoS} (\sphinxstyleemphasis{np.ndarray}) \textendash{} Density of states

\item {} 
\sphinxAtStartPar
\sphinxstylestrong{screen\_len} (\sphinxstyleemphasis{np.ndarray}) \textendash{} Screening length

\item {} 
\sphinxAtStartPar
\sphinxstylestrong{n\_imp} (\sphinxstyleemphasis{np.ndarray}) \textendash{} impurity scattering

\item {} 
\sphinxAtStartPar
\sphinxstylestrong{dielectric} (\sphinxstyleemphasis{float}) \textendash{} Dielectric constant

\end{itemize}

\item[{Returns}] \leavevmode
\sphinxAtStartPar
\sphinxstylestrong{tau} \textendash{} Electron\sphinxhyphen{}impurity lifetime

\item[{Return type}] \leavevmode
\sphinxAtStartPar
np.ndarray

\end{description}\end{quote}

\end{fulllineitems}



\subsection{ThermoElectric.tau\_screened\_coulomb}
\label{\detokenize{autosummary/ThermoElectric.tau_screened_coulomb:thermoelectric-tau-screened-coulomb}}\label{\detokenize{autosummary/ThermoElectric.tau_screened_coulomb::doc}}\index{tau\_screened\_coulomb() (in module ThermoElectric)@\spxentry{tau\_screened\_coulomb()}\spxextra{in module ThermoElectric}}

\begin{fulllineitems}
\phantomsection\label{\detokenize{autosummary/ThermoElectric.tau_screened_coulomb:ThermoElectric.tau_screened_coulomb}}\pysiglinewithargsret{\sphinxcode{\sphinxupquote{ThermoElectric.}}\sphinxbfcode{\sphinxupquote{tau\_screened\_coulomb}}}{\emph{\DUrole{n}{energy}\DUrole{p}{:}\DUrole{w}{  }\DUrole{n}{numpy.ndarray}}, \emph{\DUrole{n}{mass\_c}\DUrole{p}{:}\DUrole{w}{  }\DUrole{n}{numpy.ndarray}}, \emph{\DUrole{n}{screen\_len}\DUrole{p}{:}\DUrole{w}{  }\DUrole{n}{numpy.ndarray}}, \emph{\DUrole{n}{n\_imp}\DUrole{p}{:}\DUrole{w}{  }\DUrole{n}{numpy.ndarray}}, \emph{\DUrole{n}{dielectric}\DUrole{p}{:}\DUrole{w}{  }\DUrole{n}{float}}}{{ $\rightarrow$ numpy.ndarray}}
\sphinxAtStartPar
Electron\sphinxhyphen{}ion scattering rate — Brook\sphinxhyphen{}Herring model

\sphinxAtStartPar
Note that for highly doped semiconductors, screen length plays a significant role,
therefor should be computed carefully. Highly suggest to use following matlab file “Fermi.m”
from: \sphinxurl{https://www.mathworks.com/matlabcentral/fileexchange/13616-fermi}

\sphinxAtStartPar
If committed to use python, the package “dfint” works with python2
pip install fdint
\begin{quote}\begin{description}
\item[{Parameters}] \leavevmode\begin{itemize}
\item {} 
\sphinxAtStartPar
\sphinxstylestrong{energy} (\sphinxstyleemphasis{np.ndarray}) \textendash{} Energy range

\item {} 
\sphinxAtStartPar
\sphinxstylestrong{mass\_c} (\sphinxstyleemphasis{np.ndarray}) \textendash{} Conduction band effective mass

\item {} 
\sphinxAtStartPar
\sphinxstylestrong{screen\_len} (\sphinxstyleemphasis{np.ndarray}) \textendash{} Screening length

\item {} 
\sphinxAtStartPar
\sphinxstylestrong{n\_imp} (\sphinxstyleemphasis{np.ndarray}) \textendash{} impurity scattering

\item {} 
\sphinxAtStartPar
\sphinxstylestrong{dielectric} (\sphinxstyleemphasis{float}) \textendash{} Dielectric constant

\end{itemize}

\item[{Returns}] \leavevmode
\sphinxAtStartPar
\sphinxstylestrong{tau} \textendash{} Electron\sphinxhyphen{}impurity lifetime

\item[{Return type}] \leavevmode
\sphinxAtStartPar
np.ndarray

\end{description}\end{quote}

\end{fulllineitems}



\subsection{ThermoElectric.tau\_unscreened\_coulomb}
\label{\detokenize{autosummary/ThermoElectric.tau_unscreened_coulomb:thermoelectric-tau-unscreened-coulomb}}\label{\detokenize{autosummary/ThermoElectric.tau_unscreened_coulomb::doc}}\index{tau\_unscreened\_coulomb() (in module ThermoElectric)@\spxentry{tau\_unscreened\_coulomb()}\spxextra{in module ThermoElectric}}

\begin{fulllineitems}
\phantomsection\label{\detokenize{autosummary/ThermoElectric.tau_unscreened_coulomb:ThermoElectric.tau_unscreened_coulomb}}\pysiglinewithargsret{\sphinxcode{\sphinxupquote{ThermoElectric.}}\sphinxbfcode{\sphinxupquote{tau\_unscreened\_coulomb}}}{\emph{\DUrole{n}{energy}\DUrole{p}{:}\DUrole{w}{  }\DUrole{n}{numpy.ndarray}}, \emph{\DUrole{n}{mass\_c}\DUrole{p}{:}\DUrole{w}{  }\DUrole{n}{numpy.ndarray}}, \emph{\DUrole{n}{n\_imp}\DUrole{p}{:}\DUrole{w}{  }\DUrole{n}{numpy.ndarray}}, \emph{\DUrole{n}{dielectric}\DUrole{p}{:}\DUrole{w}{  }\DUrole{n}{float}}}{{ $\rightarrow$ numpy.ndarray}}
\sphinxAtStartPar
Electron\sphinxhyphen{}ion scattering rate for shallow dopants \textasciitilde{}10\textasciicircum{}18 1/cm\textasciicircum{}3
(no screening effect is considered)
\begin{quote}\begin{description}
\item[{Parameters}] \leavevmode\begin{itemize}
\item {} 
\sphinxAtStartPar
\sphinxstylestrong{energy} (\sphinxstyleemphasis{np.ndarray}) \textendash{} Energy range

\item {} 
\sphinxAtStartPar
\sphinxstylestrong{mass\_c} (\sphinxstyleemphasis{np.ndarray}) \textendash{} Conduction band effective mass

\item {} 
\sphinxAtStartPar
\sphinxstylestrong{n\_imp} (\sphinxstyleemphasis{np.ndarray}) \textendash{} impurity scattering

\item {} 
\sphinxAtStartPar
\sphinxstylestrong{dielectric} (\sphinxstyleemphasis{float}) \textendash{} Dielectric constant

\end{itemize}

\item[{Returns}] \leavevmode
\sphinxAtStartPar
\sphinxstylestrong{tau} \textendash{} Electron\sphinxhyphen{}impurity lifetime

\item[{Return type}] \leavevmode
\sphinxAtStartPar
np.ndarray

\end{description}\end{quote}

\end{fulllineitems}



\subsection{ThermoElectric.tau\_2d\_cylinder}
\label{\detokenize{autosummary/ThermoElectric.tau_2d_cylinder:thermoelectric-tau-2d-cylinder}}\label{\detokenize{autosummary/ThermoElectric.tau_2d_cylinder::doc}}\index{tau\_2d\_cylinder() (in module ThermoElectric)@\spxentry{tau\_2d\_cylinder()}\spxextra{in module ThermoElectric}}

\begin{fulllineitems}
\phantomsection\label{\detokenize{autosummary/ThermoElectric.tau_2d_cylinder:ThermoElectric.tau_2d_cylinder}}\pysiglinewithargsret{\sphinxcode{\sphinxupquote{ThermoElectric.}}\sphinxbfcode{\sphinxupquote{tau\_2d\_cylinder}}}{\emph{\DUrole{n}{energy}\DUrole{p}{:}\DUrole{w}{  }\DUrole{n}{numpy.ndarray}}, \emph{\DUrole{n}{num\_kpoints}\DUrole{p}{:}\DUrole{w}{  }\DUrole{n}{numpy.ndarray}}, \emph{\DUrole{n}{Uo}\DUrole{p}{:}\DUrole{w}{  }\DUrole{n}{float}}, \emph{\DUrole{n}{relative\_mass}\DUrole{p}{:}\DUrole{w}{  }\DUrole{n}{numpy.ndarray}}, \emph{\DUrole{n}{volume\_frac}\DUrole{p}{:}\DUrole{w}{  }\DUrole{n}{float}}, \emph{\DUrole{n}{valley}\DUrole{p}{:}\DUrole{w}{  }\DUrole{n}{numpy.ndarray}}, \emph{\DUrole{n}{dk\_len}\DUrole{p}{:}\DUrole{w}{  }\DUrole{n}{float}}, \emph{\DUrole{n}{ro}\DUrole{p}{:}\DUrole{w}{  }\DUrole{n}{numpy.ndarray}}, \emph{\DUrole{n}{lattice\_parameter}\DUrole{p}{:}\DUrole{w}{  }\DUrole{n}{float}}, \emph{\DUrole{n}{n\_sample}\DUrole{o}{=}\DUrole{default_value}{2000}}}{{ $\rightarrow$ numpy.ndarray}}
\sphinxAtStartPar
A fast algorithm that uses Fermi’s golden rule to compute the energy dependent electron scattering rate
from cylindrical nano\sphinxhyphen{}particles or nano\sphinxhyphen{}scale pores infinitely extended perpendicular to the current.
\begin{quote}\begin{description}
\item[{Parameters}] \leavevmode\begin{itemize}
\item {} 
\sphinxAtStartPar
\sphinxstylestrong{energy} (\sphinxstyleemphasis{np.ndarray}) \textendash{} Energy range

\item {} 
\sphinxAtStartPar
\sphinxstylestrong{num\_kpoints} (\sphinxstyleemphasis{np.ndarray}) \textendash{} Number of kpoints in each direction

\item {} 
\sphinxAtStartPar
\sphinxstylestrong{Uo} (\sphinxstyleemphasis{float}) \textendash{} Barrier height

\item {} 
\sphinxAtStartPar
\sphinxstylestrong{relative\_mass} (\sphinxstyleemphasis{np.ndarray}) \textendash{} Relative mass of electron

\item {} 
\sphinxAtStartPar
\sphinxstylestrong{volume\_frac} (\sphinxstyleemphasis{float}) \textendash{} Defects volume fraction

\item {} 
\sphinxAtStartPar
\sphinxstylestrong{valley} (\sphinxstyleemphasis{np.ndarray}) \textendash{} Conduction band valley indices

\item {} 
\sphinxAtStartPar
\sphinxstylestrong{dk\_len} (\sphinxstyleemphasis{float}) \textendash{} Sample size

\item {} 
\sphinxAtStartPar
\sphinxstylestrong{ro} (\sphinxstyleemphasis{np.ndarray}) \textendash{} Cylinder radius

\item {} 
\sphinxAtStartPar
\sphinxstylestrong{lattice\_parameter} (\sphinxstyleemphasis{float}) \textendash{} lattice parameter

\item {} 
\sphinxAtStartPar
\sphinxstylestrong{n\_sample} (\sphinxstyleemphasis{int}) \textendash{} Mesh sample size

\end{itemize}

\item[{Returns}] \leavevmode
\sphinxAtStartPar
\sphinxstylestrong{tau\_cylinder} \textendash{} Electron\sphinxhyphen{}defect lifetime

\item[{Return type}] \leavevmode
\sphinxAtStartPar
np.ndarray

\end{description}\end{quote}

\end{fulllineitems}



\subsection{ThermoElectric.tau3D\_spherical}
\label{\detokenize{autosummary/ThermoElectric.tau3D_spherical:thermoelectric-tau3d-spherical}}\label{\detokenize{autosummary/ThermoElectric.tau3D_spherical::doc}}\index{tau3D\_spherical() (in module ThermoElectric)@\spxentry{tau3D\_spherical()}\spxextra{in module ThermoElectric}}

\begin{fulllineitems}
\phantomsection\label{\detokenize{autosummary/ThermoElectric.tau3D_spherical:ThermoElectric.tau3D_spherical}}\pysiglinewithargsret{\sphinxcode{\sphinxupquote{ThermoElectric.}}\sphinxbfcode{\sphinxupquote{tau3D\_spherical}}}{\emph{\DUrole{n}{num\_kpoints}\DUrole{p}{:}\DUrole{w}{  }\DUrole{n}{numpy.ndarray}}, \emph{\DUrole{n}{Uo}\DUrole{p}{:}\DUrole{w}{  }\DUrole{n}{float}}, \emph{\DUrole{n}{relative\_mass}\DUrole{p}{:}\DUrole{w}{  }\DUrole{n}{numpy.ndarray}}, \emph{\DUrole{n}{volume\_frac}\DUrole{p}{:}\DUrole{w}{  }\DUrole{n}{float}}, \emph{\DUrole{n}{valley}\DUrole{p}{:}\DUrole{w}{  }\DUrole{n}{numpy.ndarray}}, \emph{\DUrole{n}{dk\_len}\DUrole{p}{:}\DUrole{w}{  }\DUrole{n}{float}}, \emph{\DUrole{n}{ro}\DUrole{p}{:}\DUrole{w}{  }\DUrole{n}{numpy.ndarray}}, \emph{\DUrole{n}{lattice\_parameter}\DUrole{p}{:}\DUrole{w}{  }\DUrole{n}{float}}, \emph{\DUrole{n}{n\_sample}\DUrole{o}{=}\DUrole{default_value}{32}}}{{ $\rightarrow$ numpy.ndarray}}
\sphinxAtStartPar
A fast algorithm that uses Fermi’s golden rule to compute the energy dependent electron scattering rate
from spherical nano\sphinxhyphen{}particles or nano\sphinxhyphen{}scale pores.
\begin{quote}\begin{description}
\item[{Parameters}] \leavevmode\begin{itemize}
\item {} 
\sphinxAtStartPar
\sphinxstylestrong{num\_kpoints} (\sphinxstyleemphasis{np.ndarray}) \textendash{} Number of kpoints in each direction

\item {} 
\sphinxAtStartPar
\sphinxstylestrong{Uo} (\sphinxstyleemphasis{float}) \textendash{} Barrier height

\item {} 
\sphinxAtStartPar
\sphinxstylestrong{relative\_mass} (\sphinxstyleemphasis{np.ndarray}) \textendash{} Relative mass of electron

\item {} 
\sphinxAtStartPar
\sphinxstylestrong{volume\_frac} (\sphinxstyleemphasis{float}) \textendash{} Defects volume fraction

\item {} 
\sphinxAtStartPar
\sphinxstylestrong{valley} (\sphinxstyleemphasis{np.ndarray}) \textendash{} Conduction band valley indices

\item {} 
\sphinxAtStartPar
\sphinxstylestrong{dk\_len} (\sphinxstyleemphasis{float}) \textendash{} Sample size

\item {} 
\sphinxAtStartPar
\sphinxstylestrong{ro} (\sphinxstyleemphasis{np.ndarray}) \textendash{} Cylinder radius

\item {} 
\sphinxAtStartPar
\sphinxstylestrong{lattice\_parameter} (\sphinxstyleemphasis{float}) \textendash{} lattice parameter

\item {} 
\sphinxAtStartPar
\sphinxstylestrong{n\_sample} (\sphinxstyleemphasis{int}) \textendash{} Mesh sample size

\end{itemize}

\item[{Returns}] \leavevmode
\sphinxAtStartPar
\sphinxstylestrong{tau} \textendash{} Electron\sphinxhyphen{}defect lifetime

\item[{Return type}] \leavevmode
\sphinxAtStartPar
np.ndarray

\end{description}\end{quote}

\end{fulllineitems}



\subsection{ThermoElectric.electrical\_properties}
\label{\detokenize{autosummary/ThermoElectric.electrical_properties:thermoelectric-electrical-properties}}\label{\detokenize{autosummary/ThermoElectric.electrical_properties::doc}}\index{electrical\_properties() (in module ThermoElectric)@\spxentry{electrical\_properties()}\spxextra{in module ThermoElectric}}

\begin{fulllineitems}
\phantomsection\label{\detokenize{autosummary/ThermoElectric.electrical_properties:ThermoElectric.electrical_properties}}\pysiglinewithargsret{\sphinxcode{\sphinxupquote{ThermoElectric.}}\sphinxbfcode{\sphinxupquote{electrical\_properties}}}{\emph{\DUrole{n}{E}\DUrole{p}{:}\DUrole{w}{  }\DUrole{n}{numpy.ndarray}}, \emph{\DUrole{n}{DoS}\DUrole{p}{:}\DUrole{w}{  }\DUrole{n}{numpy.ndarray}}, \emph{\DUrole{n}{vg}\DUrole{p}{:}\DUrole{w}{  }\DUrole{n}{numpy.ndarray}}, \emph{\DUrole{n}{Ef}\DUrole{p}{:}\DUrole{w}{  }\DUrole{n}{numpy.ndarray}}, \emph{\DUrole{n}{dfdE}\DUrole{p}{:}\DUrole{w}{  }\DUrole{n}{numpy.ndarray}}, \emph{\DUrole{n}{temp}\DUrole{p}{:}\DUrole{w}{  }\DUrole{n}{numpy.ndarray}}, \emph{\DUrole{n}{tau}\DUrole{p}{:}\DUrole{w}{  }\DUrole{n}{numpy.ndarray}}}{{ $\rightarrow$ dict}}
\sphinxAtStartPar
This function returns electronic properties.
Good references on this topic are:
“Near\sphinxhyphen{}equilibrium Transport: Fundamentals And Applications” by Changwook Jeong and Mark S. Lundstrom
“Nanoscale Energy Transport and Conversion: A Parallel Treatment of Electrons, Molecules, Phonons, and Photons” by Gang Chen.
\begin{quote}\begin{description}
\item[{Parameters}] \leavevmode\begin{itemize}
\item {} 
\sphinxAtStartPar
\sphinxstylestrong{E} (\sphinxstyleemphasis{np.ndarray}) \textendash{} Energy range

\item {} 
\sphinxAtStartPar
\sphinxstylestrong{DoS} (\sphinxstyleemphasis{np.ndarray}) \textendash{} Electron density of state

\item {} 
\sphinxAtStartPar
\sphinxstylestrong{vg} (\sphinxstyleemphasis{np.ndarray}) \textendash{} Group velocity

\item {} 
\sphinxAtStartPar
\sphinxstylestrong{Ef} (\sphinxstyleemphasis{np.ndarray}) \textendash{} Fermi level

\item {} 
\sphinxAtStartPar
\sphinxstylestrong{dfdE} (\sphinxstyleemphasis{np.ndarray}) \textendash{} Fermi window

\item {} 
\sphinxAtStartPar
\sphinxstylestrong{temp} (\sphinxstyleemphasis{np.ndarray}) \textendash{} Temperature range

\item {} 
\sphinxAtStartPar
\sphinxstylestrong{tau} (\sphinxstyleemphasis{np.ndarray}) \textendash{} Lifetime

\end{itemize}

\item[{Returns}] \leavevmode
\sphinxAtStartPar
\sphinxstylestrong{coefficients} \textendash{} Linear BTE prediction of electrical properties

\item[{Return type}] \leavevmode
\sphinxAtStartPar
dict

\end{description}\end{quote}

\end{fulllineitems}



\subsection{ThermoElectric.filtering\_effect}
\label{\detokenize{autosummary/ThermoElectric.filtering_effect:thermoelectric-filtering-effect}}\label{\detokenize{autosummary/ThermoElectric.filtering_effect::doc}}\index{filtering\_effect() (in module ThermoElectric)@\spxentry{filtering\_effect()}\spxextra{in module ThermoElectric}}

\begin{fulllineitems}
\phantomsection\label{\detokenize{autosummary/ThermoElectric.filtering_effect:ThermoElectric.filtering_effect}}\pysiglinewithargsret{\sphinxcode{\sphinxupquote{ThermoElectric.}}\sphinxbfcode{\sphinxupquote{filtering\_effect}}}{\emph{\DUrole{n}{Uo}}, \emph{\DUrole{n}{E}}, \emph{\DUrole{n}{DoS}}, \emph{\DUrole{n}{vg}}, \emph{\DUrole{n}{Ef}}, \emph{\DUrole{n}{dfdE}}, \emph{\DUrole{n}{temp}}, \emph{\DUrole{n}{tau\_bulk}}}{}
\sphinxAtStartPar
This function returns electric properties for the ideal filtering —  where all the electrons up to a cutoff energy
level of Uo are completely hindered.
\begin{quote}\begin{description}
\item[{Parameters}] \leavevmode\begin{itemize}
\item {} 
\sphinxAtStartPar
\sphinxstylestrong{Uo} (\sphinxstyleemphasis{np.ndarray}) \textendash{} Barrier height

\item {} 
\sphinxAtStartPar
\sphinxstylestrong{E} (\sphinxstyleemphasis{np.ndarray}) \textendash{} Energy range

\item {} 
\sphinxAtStartPar
\sphinxstylestrong{DoS} (\sphinxstyleemphasis{np.ndarray}) \textendash{} Electron density of state

\item {} 
\sphinxAtStartPar
\sphinxstylestrong{vg} (\sphinxstyleemphasis{np.ndarray}) \textendash{} Group velocity

\item {} 
\sphinxAtStartPar
\sphinxstylestrong{Ef} (\sphinxstyleemphasis{np.ndarray}) \textendash{} Fermi level

\item {} 
\sphinxAtStartPar
\sphinxstylestrong{dfdE} (\sphinxstyleemphasis{np.ndarray}) \textendash{} Fermi window

\item {} 
\sphinxAtStartPar
\sphinxstylestrong{temp} (\sphinxstyleemphasis{np.ndarray}) \textendash{} Temperature range

\item {} 
\sphinxAtStartPar
\sphinxstylestrong{tau\_bulk} (\sphinxstyleemphasis{np.ndarray}) \textendash{} Lifetime in bulk material

\end{itemize}

\item[{Returns}] \leavevmode
\sphinxAtStartPar
\sphinxstylestrong{output} \textendash{} Linear BTE prediction of electrical properties

\item[{Return type}] \leavevmode
\sphinxAtStartPar
dict

\end{description}\end{quote}

\end{fulllineitems}



\subsection{ThermoElectric.phenomenological}
\label{\detokenize{autosummary/ThermoElectric.phenomenological:thermoelectric-phenomenological}}\label{\detokenize{autosummary/ThermoElectric.phenomenological::doc}}\index{phenomenological() (in module ThermoElectric)@\spxentry{phenomenological()}\spxextra{in module ThermoElectric}}

\begin{fulllineitems}
\phantomsection\label{\detokenize{autosummary/ThermoElectric.phenomenological:ThermoElectric.phenomenological}}\pysiglinewithargsret{\sphinxcode{\sphinxupquote{ThermoElectric.}}\sphinxbfcode{\sphinxupquote{phenomenological}}}{\emph{\DUrole{n}{Uo}}, \emph{\DUrole{n}{tau\_o}}, \emph{\DUrole{n}{E}}, \emph{\DUrole{n}{DoS}}, \emph{\DUrole{n}{vg}}, \emph{\DUrole{n}{Ef}}, \emph{\DUrole{n}{dfdE}}, \emph{\DUrole{n}{temp}}, \emph{\DUrole{n}{tau\_bulk}}}{}
\sphinxAtStartPar
This function returns electric properties for the phenomenological filtering —  where a frequency independent
lifetime of tau\_o is imposed to all the electrons up to a cutoff energy level of Uo
\begin{quote}\begin{description}
\item[{Parameters}] \leavevmode\begin{itemize}
\item {} 
\sphinxAtStartPar
\sphinxstylestrong{Uo} (\sphinxstyleemphasis{np.ndarray}) \textendash{} Barrier height

\item {} 
\sphinxAtStartPar
\sphinxstylestrong{tau\_o} (\sphinxstyleemphasis{np.ndarray}) \textendash{} Phenomenological lifetime

\item {} 
\sphinxAtStartPar
\sphinxstylestrong{E} (\sphinxstyleemphasis{np.ndarray}) \textendash{} Energy range

\item {} 
\sphinxAtStartPar
\sphinxstylestrong{DoS} (\sphinxstyleemphasis{np.ndarray}) \textendash{} Electron density of state

\item {} 
\sphinxAtStartPar
\sphinxstylestrong{vg} (\sphinxstyleemphasis{np.ndarray}) \textendash{} Group velocity

\item {} 
\sphinxAtStartPar
\sphinxstylestrong{Ef} (\sphinxstyleemphasis{np.ndarray}) \textendash{} Fermi level

\item {} 
\sphinxAtStartPar
\sphinxstylestrong{dfdE} (\sphinxstyleemphasis{np.ndarray}) \textendash{} Fermi window

\item {} 
\sphinxAtStartPar
\sphinxstylestrong{temp} (\sphinxstyleemphasis{np.ndarray}) \textendash{} Temperature range

\item {} 
\sphinxAtStartPar
\sphinxstylestrong{tau\_bulk} (\sphinxstyleemphasis{np.ndarray}) \textendash{} Lifetime in bulk material

\end{itemize}

\item[{Returns}] \leavevmode
\sphinxAtStartPar
\sphinxstylestrong{output} \textendash{} Linear BTE prediction of electrical properties

\item[{Return type}] \leavevmode
\sphinxAtStartPar
dict

\end{description}\end{quote}

\end{fulllineitems}




\renewcommand{\indexname}{Index}
\printindex
\end{document}